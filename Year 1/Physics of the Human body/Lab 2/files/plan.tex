\subsection*{Hypothesis}
	The aim of the experiment is to explore the effect of fatigue, whether emotional (giving up) or muscular (unable to lift), on blood oxygenation and heart rate during anaerobic exercise.  

	During aerobic exercise we expect blood oxygenation to reduce as oxygen is moved to the muscles. We infer that anaerobic exercises have a much higher oxygen requirement, as fatigue is reached considerably quicker. Thus, the hypothesis is that blood oxygenation should be higher during early work-out and lower as fatigue is reached.

	Given that higher heart rates are associated with aerobic exercising, we explore whether we reach similar numbers during an anaerobic exercise -- especially close to fatigue. Despite the aforementioned higher oxygen requirement, a lower pulse rate is expected.

\subsection*{Experimental design}
	The study has been developed by only one student and to be performed by only one student, and this is taken into consideration in the design. The task, consequently, consists of performing deadlifts to ``exhaustion'' within 5 to 8 repetitions, such as to remain in what is usually considered anaerobic. The exercise is repeated for 6 sets with a 1 minute interval between them.
	We are using a finger-attached pulse oximeter, therefore a whole-body exercise is chosen so as to both push the individual harder and avoid blood pooling away from the fingers.

	Since anaerobic exercises are done in short bursts, blood oxygenation and heart rate are measured while resting, upon completion of each set, and before initiating the subsequent set. These are performed using a pulse oximeter, and we compare the heart rate with a manually measured one against the carotid artery. 

	Determining the appropriate weight to use and whether fatigue has been reached are very individual, hence it is preferable to have a subject who is familiar with the exercise and the gym generally, so as to reach muscular, not emotional, fatigue.

	Finally, in warming up with aerobic exercises, we establish a control against which we may compare anaerobic fatigue heart rates.