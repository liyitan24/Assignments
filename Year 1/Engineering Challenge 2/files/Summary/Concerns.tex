The scaling of the project is done as conservatively as possible, considering the aforementioned factors. It would be accompanied by long-term a vaccination campaign\cite{GomezM.N.Altet1993Reot}, but considering low vaccination adoption following campaigns even in highly developed nations \cite{BjorkmanIngeborg2013TSAv} , we expect to cover similar numbers short-term -- around 60\% of the population, approximately 25 million Ugandans -- and another 5 million people in neighbouring countries. 

Hence, we have chosen to utilise bioreactors with 30,000 litre capacities. These are the most economically sustainable option restricting over-production and excess use of resources in a developing economy. Moreover, by implementing realistic annual functioning times of 40 weeks per year, we can expect 20 batches of vaccines each year, allowing for immunisation coverage of the aforementioned 30 million people.  

One way to ensure that the vaccine reaches the most vulnerable groups is through mobile vaccination posts, a concept widely spread in another developing nation, Brazil. Supported by campaigns, these have a much larger reach than expecting the populace to travel to larger urban centres. 

A large point of contention for our plant is the immediate area surrounding it – both in terms of local labour and the possible displacement caused by the required infrastructure. By being such an important employer, there is motivation to get involved in both early education and technical training of adults. Not only is there a direct benefit from more skilled labour, but there is a larger social approval, as well as the  improved outcomes from better education \cite{OreopoulosPhilip2013MCWI}.