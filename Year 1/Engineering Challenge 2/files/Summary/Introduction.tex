\subsection{Aim}
In Uganda 5000 deaths each year are tuberculosis (TB)-related and is increasing annually, with a prevalence of 253 per 100,000 people, compared to 159 per 100,000 in 2015 \cite{Olle-GoigJaimeE.2004DfMi}.

As a team, we were given the brief to develop the bioreactor to produce a new protein-extraction based vaccine. We were divided into sub-teams of mixing, heating and growth-monitoring to design a small-scale model to mimic an industrial scale reactor.  

Using the results of our 24hr run, we aimed to evaluate the specific parameters required for large-scale reactors. We wanted to determine the size and number of reactors needed, and how many doses of vaccines can be produced within the most cost-efficient conditions.  

Specifically, we worked on the ‘Testing’ stage of the design cycle; proposing a prototype, testing it and evaluating where changes should occur and whether it is a feasible solution with the potential to be manufactured.

\subsection{Country profile}
Uganda is a land-locked, Sub-Saharan LEDC (Less economically developed country) with a GDP value of 32.8 billion since 2018 \cite{WB:2019Uganda}. Despite promising pioneering towards the secondary sector, generating more revenue than the previous agricultural sector, 19.7\% (Est. 2017) of the population live below the poverty line. Moreover, the HDI (Human development index) of the country, currently standing at 0.528, accounts for the low life expectancy of 55 years and is potentially caused by health expenditure being as low as 7.2\% GDP (Gross Domestic product) in 2014 \cite{BigstenArne2001IUae}.

 Developing economy as is Uganda’s, the integration and consideration of themes such as culture, ethics, sustainability and safety are indispensable.  

From a cultural stand-point, the establishment of a vaccine manufacturing facility and the construction of a power plant will have various impacts on the lives of local people.  

Mbarara, the location of the powerplant, is the largest milk producer of the area with 90\% of the local population keeping cows to generate income; however, to build the necessary infrastructure, local farmers may need to sell their land, and the government will have to compensate with the provision of employment to those impacted. 

Uganda is a widely religious country with Christianity and the Muslim faith dominating statistics. In regards to any religion-based opposition that may arise, the World Bank has advised complete adoption of vaccination in the country. Education on the importance of vaccines will be given to the public, however, little to no lobbying must be done concerning religion or any anti-vaccination movements, given the sustained increase in vaccination incidence in the country. \cite{WB:2016, WB:2018} 

Another potential issue is the political situation in Uganda. Indeed, President Yoweri Museveni's growing authoritarianism and the country’s weak institutions are multiplying Uganda’s challenges. Economic stagnation and an uncertain political succession are causing an increase in the risk of conflict at the local level. The state’s repression of political opposition and privatisation of public sectors are increasing discontent within the marginalized population, impacting public response to vaccination campaign that is impart and government-funded. \cite{CrisisGroup2012}. 

On the other hand, recurrent government health expenditure in Uganda reflects its policies to prioritize health, with thrice increased spending on hospital services in nominal terms from 27 billion in 1997 to 76 billion in 2008 \cite{MugishaFrederick2010Twed}. 

This health-investment mentality from the government provides a spring-board for the implementation of such a project which rendering it increasingly feasible culturally and economically. 