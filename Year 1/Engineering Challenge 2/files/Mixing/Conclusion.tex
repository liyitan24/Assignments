In this lab, four comparative experiments were conducted to determine the optimal bioreactor set up.
The impacts of rotational speed, impeller type, baffles, and impeller numbers on liquid mixing time were explored by operating the mixing system under various conditions. Data were recorded throughout all experiments and plotted in figures.
The results were consistent with our hypotheses that Pitched-blade would be the best choice due to its size, baffles would assist mixing by reducing vortex, and multiple impellers would work more efficiently than individual ones.
Therefore, the optimal setting is decided to be operating the mixing equipment with 1 Rushton turbine and 1 pitched-blade at 280rpm under baffled condition. 

For future work, baffles of the precise dimension could be made via 3D printing to optimize their vortex elimination function.
A cylindrical vessel could be used in real culturing condition.
A monitoring system should be developed to replace the manual timing and supervising of mixing.
In this lab, blue dye was added above the liquid level; however, when culturing cells, the position of feed addition point should be under liquid level and should ensure optimal distribution. 