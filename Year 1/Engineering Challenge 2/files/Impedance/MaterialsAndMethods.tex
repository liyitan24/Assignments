\subsubsection{Materials}
\begin{itemize}
	\item Combined amplifier, filter and driving circuit
	\item Function generator and oscilloscope. NI myDAQ used for both.
	\item Phosphate buffer solution (PBS)
	\item Activated yeast (\textit{Saccharomyces cerevisiae}), a 20g/l and a 50 g/l solution.
\end{itemize}

\subsubsection{Methods} \label{sec:Impedance-Method}
\begin{enumerate}
	\item Attach the probes to the cup, filled with 300 ml of each solution at a time.
	\item For each solution measure the voltage output over the frequency range from 1 Hz to 20 kHz, using different gain ratios. Use this to establish the appropriate amplification.
	\item Sweep the frequency range stated in step 2 by keeping function generator's peak-to-peak voltage input fixed at 10 V.
	\item Finally, we generate the calibration curve using a 50g/l yeast solution, and repeating step 3 for a series of consecutive dilutions (expressed as \% from the original): 100\%, 20\%, 10\%, 4\%, 2\%, 1\%. 
\end{enumerate}
\paragraph{Variables}
\begin{itemize}
	\item Control:
		\begin{itemize}
			\item Output voltage on function generator: 10V.
			\item Probes 180º from each other. Sense leads 30º from the probe and 180º from each other. See Figure \ref{fig:ElectrodeOrientation}.
			\item Amplifier gain $\frac{15k}{150} = 100$.
			\item Solution volume: 300 ml
		\end{itemize}
	\item Independent: Frequency.
	\item Dependent: Potential difference
\end{itemize}
\begin{figure}[h]
	\newcommand{\Radius}{3}
	\newcommand{\RadPow}{0.5}
	\newcommand{\RadVolt}{0.5}
	\centering
	\begin{circuitikz}[scale=0.75]
		\draw (0,0) circle (\Radius cm);
		%Driving electrodes
		\draw [dashed,cyan] (0,-\Radius) node{*} -- (0,\Radius) node{*};
		\node at (0.3,-\Radius-0.3) {-$V_x$};
		\node at (0.3,\Radius+0.3) {$V_x$};
		\draw [cyan] (0,\Radius) -- ++ (0,\RadPow)
			-- ++ ({-\Radius*1.3},0)
			-- ++ (0,{-\Radius}) 
			++ (0,0) to[vsourcesin,a=\SI{10}{\volt},color=cyan] ++ (0,-2*\RadPow)
			-- ++ (0,{-\Radius})
			-- ++ ({\Radius*1.3},0)
			-- ++ (0,{\RadPow})
		;
		% Sensor electrodes
		\draw[dashed, red] ({\Radius*cos(60)},{-\Radius*sin(60)}) node{*} -- ({-\Radius*cos(60)},{\Radius*sin(60)}) node{*} coordinate (a);
		\draw [red]
			(a) -- ++ (-{(2*\RadVolt*0.8)*cos(60)}, {(2*\RadVolt*0.8)*sin(60)}) coordinate (a)
			-- ++ (0,0) arc (-60:120:0.2)
			-- ++ ({1.3*\Radius*cos(30)},{1.3*\Radius*sin(30)})
			-- ++ ({(\Radius+\RadVolt)*cos(60)},{-(\Radius+\RadVolt)*sin(60)})
			++ ({\RadVolt*cos(60)},{-\RadVolt*sin(60)}) circle (\RadVolt) node{V} ++ ({\RadVolt*cos(60)},{-\RadVolt*sin(60)})
			-- ++ ({(\Radius+\RadVolt)*cos(60)},{-(\Radius+\RadVolt)*sin(60)})
			-- ++ ({-1.3*\Radius*cos(30)},-{1.3*\Radius*sin(30)})
			-- ++ ({(-2*\RadVolt*0.8)*cos(60)}, {(2*\RadVolt*0.8)*sin(60)})
		;
		%
		\draw [<->] (0,\Radius*0.5) arc (90:120:\Radius*0.5) node[pos=0.5,anchor=south]{30º};
		\matrix [draw, matrix of math nodes] at (current bounding box.north east) {
    		 \color{cyan}* & \text{Driving electrodes} \\
			 \color{red}* & \text{Sensing electrodes} \\
	 	};
	\end{circuitikz}
	\caption{Orientation of electrodes}
	\label{fig:ElectrodeOrientation}
\end{figure}

\paragraph{Considerations}

Due to a limited power input into the amplifier, a gain ratio of 1000 led to \textit{gain saturation}, where we no longer observed the expected amplification.
With gain of 10, the maxima appeared in different frequencies as to those of 100 and 1000, though we are unsure why this would happen.

Although we defined probe orientation, there are two factors we are disregarding: the cup is not perfectly cylindrical, and the leads are not straight -- so the distance between electrodes is smaller than what the experimental setup implies.