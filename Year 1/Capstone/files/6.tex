Hi, John.

Applying to study biomedical engineering is a fantastic idea! There is an interesting balance of practical work, introduction to real life applications and purely theoretical. Our first engineering challenge, for example, was heavily focused on introducing imaging techniques and making us reflect not only on the techniques themselves, but how practical their use was in a poorer nation. Ultimately getting as far as whether the best solution was even technical in nature. That reflection is probably what I will find most useful for the rest of my life, but  the technical and mathematical sides were supported in other classes -- including later on having a full module on introduction to medical imaging.

Our second challenge was considerably more driven by our knowledge in building electronic circuits as taught in two of our modules. Largely the purpose of it seemed to have been to interact with other engineering fields, trying to superficially understand what they are doing in order to create one cohesive project. Ultimately working with people with varying degrees of interest was (and is) an important skill to have, and it certainly seems like that was the learning objective.

We have also had a large focus on the human body in particular -- even in our Mechanics and Material class, where we discuss standard physics, there is an ongoing conversation about biological materials, for example. An important skill that is used in this class, as well as Physics of the Human Body and Cardiac Engineering, is to break a biological system into a simplified mathematical/physical model. Although these may not be areas I plan to working in, the principle being taught is universal.

By and large the purpose of our first year was to introduce a few concepts, some of which are of very little interest to my pursuit in prosthetics, but it overwhelmingly also gave me really important tools which are completely subject independent.