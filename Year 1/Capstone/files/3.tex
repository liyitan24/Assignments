\subsection*{Part A}
\subsubsection*{i}
Convert range of \SIrange{1}{100}{\micro V} $\implies$ \SIrange{0}{5}{V}
\begin{equation}
    \text{Amplification} = \frac{5-0}{10^{-4}-10^{-6}} = 50505
\end{equation}
Leading to a range of 50505 $\cdot$ (\SIrange{1}{100}{\micro V}) = \SIrange{0.050505}{5.050505}{V}. Thus requiring a DC offset of \SI{0.050505}{V}.

We can use a band-pass filter or a combination of high-pass and low-pass in series in order to cut off any frequencies outside of the \SIrange{1}{1000}{Hz} range

\subsubsection*{ii}

\begin{figure}[h!]
    \centering
    \begin{circuitikz}
        \draw
            (0, 0) node[op amp] (opamp) {}
            (opamp.+) to[short, -o] (-2,-0.5)  node[left] {$V_{in}$}
            (opamp.-) to[align=center, R, a={$R_1$ \\ \SI{18}{\ohm} }, -] (-3, 0.5) 
            (opamp.-) to[short,*-] ++(0,1.5) coordinate (leftC)
            to[R={$R_2$ \\ \SI{910}{k\ohm}}, align=center] (leftC -| opamp.out)
            to[short,-*] (opamp.out) to[short, -] node[right] {} ++ (1,0)
        ;
        \draw (-3,0.5) node[ground] {};
        %
        \draw (1.5,0) to[C=\SI{4.7}{\micro\farad}, -] (2.5,0) to[R=\SI{33}{\ohm}] (4.5,0) to[short, -o] (6.5,0);
        \draw (2.5,0) to[R, a=\SI{36}{\kilo\ohm}, *-*] (2.5,-3);
        \draw (4.8,0) to[C, a=\SI{4.7}{\micro\farad}, *-*] (4.8,-3);
        \draw (0,-3) node[ground] {} to[short, -o] (6.5,-3);
        \draw [|->] (6.5,-2.85) -- (6.5,-0.15) node[midway, fill=white]{$V_{out}$};
    \end{circuitikz}
\end{figure}

The amplification is calculated by $1+\frac{R_2}{R_1} = 50556$, which is appropriately close to the amplification we require. Specifically opted to not use a differential amplifier with an offset, because with the components we use, the required \SI{50}{mV} offset is very unlikely to be lost within the tolerances of our circuit.

In order to calculate both cut off frequencies for our filters, we use the following expression
\begin{equation}
    f_c = \frac{1}{2\pi RC}
\end{equation}
For the selected resistors and capacitors for both filters, we have the cut-off frequencies of \SI{0.9}{Hz} and \SI{1026}{Hz}, respectively.

There are no particular requirements for the op amp itself, though we used LM358 extensively.

\subsection*{Part B}

Consider that for $\alpha, \beta, \gamma, \delta$ being above the threshold = 1, and below = 0. Finally, for at least two values = 0 our function $f = 1$ indicates brain damage.


\begin{table}[h] \centering
    \begin{tabular}{c|cccccccccccccccc}
        $\alpha$ &0&0&0&0&0&0&0&0&1&1&1&1&1&1&1&1 \\
        $\beta $ &0&0&0&0&1&1&1&1&0&0&0&0&1&1&1&1 \\
        $\gamma$ &0&0&1&1&0&0&1&1&0&0&1&1&0&0&1&1 \\
        $\delta$ &0&1&0&1&0&1&0&1&0&1&0&1&0&1&0&1 \\
        $f$      &1&1&1&1&1&1&1&0&1&1&1&0&1&0&0&0
    \end{tabular}
    \caption{Truth table where $f$ is our required function, yielding 1 when at least two indicators are below the threshold.}
 \end{table}

\begin{figure}[h!] \centering
\begin{Karnaugh}
    \contingut{1,1,1,1,1,1,1,0,1,1,1,0,1,0,0,0}
    \implicant{0}{2}{red}
    \implicant{0}{8}{purple}
    \implicant{0}{5}{blue}
    \implicantdaltbaix{0}{9}{green}
    \implicantcostats[3pt]{0}{6}{orange}
% \implicantcantons[2pt]{orange}
%    \implicantcostats{4}{14}{green}
\end{Karnaugh}
\caption{Karnaugh map. Each colour is a different grouping which leads us to the function $f$ below.}
\end{figure}
As an example, by looking at the green group, we observe that a change in $\delta$ and a change in $\alpha$ do not influence the results, which leads us to the $\overline{\gamma\beta} $ term: 
\begin{equation}
    f = \overline{\alpha\beta} + \overline{\gamma\delta} + \overline{\gamma\alpha} +\overline{\gamma\beta} + \overline{\delta\alpha} + \alpha\bar{\beta}\gamma\bar{\delta}
\end{equation}
From which we can draw the digital circuit

\begin{figure}[h!]
    \centering
    \begin{circuitikz}[scale=0.8]
        \node (a) at (0,0) {$\alpha$};
        \node (b) at (1,0) {$\beta$};
        \node (c) at (2,0) {$\gamma$};
        \node (d) at (3,0) {$\delta$};

        \node[american nand port] (and1) at (5,-1) {};
        \node[american nand port] (and2) at (5,-2.5) {};
        \node[american nand port] (and3) at (5,-4) {};
        \node[american nand port] (and4) at (5,-5.5) {};
        \node[american nand port] (and5) at (5,-7) {};
        \node[american nand port, number inputs=3] (and6) at (7,-10) {};
        \node[american or port, number inputs =6] (or1) at (9, -4.1) {};

        \draw (a) |- (and1.in 1) (b) |- (and1.in 2);
        \draw (c) |- (and2.in 1) (d) |- (and2.in 2);
        \draw (a) |- (and3.in 1) (c) |- (and3.in 2);
        \draw (b) |- (and4.in 1) (c) |- (and4.in 2);
        \draw (a) |- (and5.in 1) (d) |- (and5.in 2);
        \draw (b) |- (and6.in 2) (d) |- (and6.in 3);

        \draw (a) ++ (0,-0.7) to[short,*-*] ++ (0,-3);
        \draw (b) ++ (0,-1.3) to[short,*-*] ++ (0,-3.9);
        \draw (c) ++ (0,-2.2) to[short, *-*] ++ (0,-2.1);
        \draw (d) ++ (0,-2.8) to[short,*-*] ++ (0,-4.5);

        \draw (and3.out) to[short, -] (or1.in 3) (5.5,-4) to[short, *-] ++(0,0) |- (and6.in 1);

        \draw (and1.out) |- (or1.in 1);
        \draw (and2.out) |- (or1.in 2);
        \draw (and4.out) |- (or1.in 4);
        \draw (and5.out) |- (or1.in 5);
        \draw (and6.out) |- (or1.in 6);
        \node [anchor=west] at (or1.out) {$f$};
    \end{circuitikz}
    \caption{Digital circuit following the description of $f$.}
\end{figure}

\clearpage