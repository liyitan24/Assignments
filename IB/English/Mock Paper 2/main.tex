\documentclass[12pt,a4paper]{article}
\usepackage[letterpaper]{geometry}
\geometry{top=1.0in, bottom=1.0in, left=1.0in, right=1.0in}
\usepackage{setspace} \doublespacing
\usepackage{outlines}
\begin{document}

\section{Discuss whether or not the endings/conclusions of the two works you have studied are  satisfactory.}

The short stories which will be discussed are \textit{The elephant vanishes} by Haruki Murakami and The headstrong historian by Chimamanda Adichie.

The feeling of frustration towards Murakami's characters is unanimous and the ending of the stories often leave the reader with many unanswered questions, and although that is also true for \textit{The elephant vanishes}, unlike his other stories there is a satisfying conclusion to the story. On the other hand, Adichie's stories often deal with universal themes and are filled with relatable characters. \textit{The headstrong historian} is a wonderful example of it, although readers might find themselves very frustrated with Michael's ``incurious rigidity''.

It's difficult to discuss satisfaction in relation to Murakami's work because the term becomes somewhat illy defined. \textit{The elephant vanishes} takes place in modern day Japan and throughout the whole story it deals with themes with very congruent with modernity: balance in one's life and need to keep up appearences. The story's narrator and protagonist lacks balance in his life ``since the elephant affair'', and as he describes this affair (the vanishing of an elephant) to a newspaper editor, he completely kills the prospect of another conversation with her. In the process we learn that he is ``essentialy pragmatic'' and believes that ``things you can't sell don't count for much'', and despite not believing that unity is necessary in a \textit{kit-chin}, the better he wears this mask, the better he sells the idea of unity \textendash\ ``Probably because people are looking for a kind of unity''. The reader is not sure how to feel about it because of a seeming inadvertent parallel he drew previously, between his life and his selling point for the kitchen: ``Even the most beautifully designed item dies if it is out of balance with its surroundings''. The term satisfactory in some ways calls for a sense of catharsis, but the reader does not get it from this theme of the story. It is, actually, rather frustrating, as the reader expects some kind of development or greater message. Interestingly, however, in the very last paragraph of the story, the protagonist does seem to come to terms with the elephant's disappearence, and the reader is left to feel that perhaps accepting that the elephant is gone will allow him to find balance again. The reader is still left with many questions, as is standard from a Murakami story, but to some extent there is a satisfying conclusion.

On the other hand, the conclusion to \textit{the headstrong historian} is perhaps the polar opposite: leaving the reader both emotionally touched and vindicated. The story describes the arrival of missionaries to Nigeria in the late 1800/early 1900, and Nwemgba, a mother who sends her child to be educated by said missionaries in order to defend her family's rights in the ``white man's court''. This means that the central theme to the story is abandoning one's traditions, with its direct and indirect victims: ``even the gods had changed and no longer asked for palm wine but for gin. Had they converted, too?''. Both the child, who goes by the name of Michael and his wife, Agnes, are very unlikable characters, best described by ``incurious rigidity'' and ``limp helplessness'', respectively, and Adichie very successfully manages to enrage the reader with their, or any other missionary's, mention. Everything changes at the end of the story with Nwemgba's granddaughter, Grace, who seems to possess the fighting spirit of Nwemgba's late husband. The entire story is narrated in past-tense, and the story's ending utilises the repetition of ``It was Grace'' to tell her future, how she detests the life she built for herself, and despite multiple prizes and recognition, felt ``an odd rootlessness''; finally concluding with her adopting the name given to her by Nwemgba: Afamefuna \textendash\ ``my name will not be lost''. 

Both stories present very difficult characters which seem to be made to be disliked.


\textendash\ all three of which play a large role in why the ending is so satisfying. The latter is, in fact, the perfect description of his own life since the elephant vanished

might be an exception. An important theme to the short story is the idea of balance, seen in the almost symbiotic relationship between the elephant and its keeper, how ``the difference between them had shrunk'' when left together, the disbalance in the protagonist ``since the elephant affair'', and the parallel between the kitchen and the elephant-keeper relationship in ``Even the most beautifully designed item dies if it is out of balance with its surroundings''. \textit{The elephant affair}, of course, is the vanishing of the elephant, which when he begins to discuss it with the journalist, breaks the balance between the two of them. At this stage the protagonist also implies knowing more about the case than described before.

The protagonist has a dichotomy between internal balance, which was lost ``since the elephant affair'', and the balance in his work, becoming successful from the balance that exists with kit-chin.

There are two distinct portions of the story: the first is the introduction of the boring character, the second is his focus on this mystery and how it influenced his work and his personal live. 

As this seems to be what the author set out to achieve, in a somewhat twisted way, the ending is satisfactory, despite the incredible frustration the reader feels. \textit{The elephant vanishes} begins by setting up an incredibly boring character, who finds the need to list all articles he read before reaching the main story: ``the national news, international politics, economics'' and so much more. Indeed, Murakami succeeds in emphasising just how incredibly ordinary this character is with the wonderful image of ``brushing away my toast crumbs, [while] I studied every line of the article''. As the story progresses, the protagonist, who is also the narrator, makes himself even more uninteresting with his excessive lists, but suddenly shifts the tone of the story to a mystery

\section{Outline}
\subsection{The headstrong historian}
\begin{outline}[enumerate]
    \1 Two parallel trains of thought while reading the story
        \2 Every mention of the christianised family is terrible.
            \3 best exemplified by Michael's ``incurious rigidity'' or Mgbeke's ``limp helplessness''
        \2 There is a strong feeling of progression of time, so it's hard to know what is coming next. 
    \1 The story ends with the death of Nwamgba as the narrator tells Grace's entire story
        \2 How she realises the importance of her roots
        \2 How she detests her husband and what he represents
        \2 How she eventually becomes Afamefuna
    \1 Then finally brings it back to the present, just describing an emotional moment between her and her grandmother
        \2 It's sad, but makes the reader rejoice.
        \2 It compensates for the frustrations throughout the entire story
\end{outline}

\subsection{The elephant vanishes}
\begin{outline}[enumerate]
    \1 Story about balance and appearences
        \2 balance broken with elephant
        \2 protagonist says he doesn't believe in the need for unity, but his personal opinion only shows ``when the tie comes off''
    \1 when he tells her about the elephant, their conversation falls apart. 
    \1 He moves on, back to his life of appearences, and the more successfully he wears the mask, the better he sells the idea of unity in kit-chin
        \2 ``Things you can't sell don't count for much ''
    \1 there is a feeling he will never forget the elephant (and find balance again)
        \2 But there is a feeling of conclusion to the story. ``They will never be coming back'' gives the understanding of finality one needs to move on, and (eventually) find balance.
        \2 The parallel between balance in the kitchen and his own life
        \2 Tying up both central themes in the end of the story.
            \3 previously in the story using the kitchen as a metaphor to the elephant (``Even the most beautifully designed item dies if it is out of place''). He doesn't believe in the need for unity, but it's precisely that which is metaphorically killing him.  
\end{outline}

\end{document}