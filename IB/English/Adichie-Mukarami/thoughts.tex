\documentclass{article}
\begin{document}
The story arch in ``On seeing the 100\% perfect girl one beautiful morning april morning'' by Haruki Murakami leads the unnamed protagonist to an increasing collection of unresolved issues as the story ``progresses''. In fact, upon conclusion of the short story, the reader is left with a sense that no resolution could have been had. Literary works often show men and women struggling to resolve problems and not succeeding very well. This paper aims to discuss the extent of this idea in both the aforementioned short story by Murakami, as well as Chimamanda Adichie's ``The thing around your neck''.

Both stories are assumed to take place in the 20th century, though with very distinct atmospheres. ``The thing around your neck'' tells the story of a young african woman who moves to the USA and finds herself unhappy, facing a multitude of both external and internal problems. The character, referred to as \textit{you} is faced with the threat of sexual abuse by \textit{your} uncle, not being welcome in America and so much more. Murakami's story, on the other hand, focuses on a very short ``encounter'', better described as a passing-by, wherein a man tells the story that justifies why he was unable to talk to his ``100\% perfect girl''.

While the helplessness of Murakami's protagonist is caused by his need to live within his head, the reason why it is necessary is equal interest. The theme of superficiality within their society seems to heavily contribute to the lack of resolution that both the reader and the character feel. Upon meeting this 100\% perfect girl, the character is made to justify to ``someone'' that she wasn't in fact perfect by society's standards \textendash\ not really good-looking, with neither remarkable breasts nor eyes. The reader is left with the unresolved question of what made her so perfect, and perhaps not even the protagonist has the answer. This translates into helplessness felt by reader because his inaction is frustrating. He instead of talking to her chooses to fantasise about what he could have said, going so far as creating a ridiculous, made-up story to explain his inaction \textendash\ contributing to his otherness.

In fact, the characters in both stories suffer from an ovearching feeling of ``otherness'' which prevents any real sense of internal development or resolution. Despite Adichie's character being a very emotionally strong, with enough courage to leave the house of the aforementioned uncle, that is \textendash\ solving a problem, she is faced with yet another problem when meeting her boyfriend. The sense the reader has is that although individual problems might be solved, the otherness is, and will be, the character's burden. Although the character is painted as suffering from externalities, she is also in part to blame. ``Nobody knew where you were, because you told no one'' is both a testament to the character's actual location, but also to how she was doing as ``something would wrap around your neck''. As for Murakami's 100\% perfect boy, a similar argument can be made. There can't be a resolution because the issue he tackles are related to loneliness, not the nature of his otherness \textendash\ living a parallel life within his head.

In conclusion, the lack of resolution is an important feature of both ``On seeing the 100\% perfect girl one beautiful morning april morning'' and ``The thing around your neck'', mostly importantly because the protagonists of both stories are plagued by an unsolvable problem: their otherness. Murakami also deals with superficiality in his story, but as discussed, it only contributes to the feeling of otherness, rather than being an issue in itself, whilst in Adichie's story there is a plethora of small problems the protagonist is, in fact, able to overcome 
\end{document}