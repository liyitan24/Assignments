\documentclass[12pt,a4paper]{article}
\usepackage[letterpaper]{geometry}
\geometry{top=1.0in, bottom=1.0in, left=1.0in, right=1.0in}
\usepackage{setspace} \doublespacing

\usepackage[round, comma, authoryear]{natbib}
\bibliographystyle{unsrtnat}

\begin{document}
\section*{“In the production of knowledge, traditions of areas of knowledge offer correctives for ways of knowing.” To what extent do you agree with this statement?}

In establishing whether traditions of areas of knowledge  offer correctives for ways of knowing, we must first qualify ``tradition'': It is a standard, a framework through which scientists, artists, historians and other experts develop their respective areas of knowledge. In the modern study of natural sciences this is the scientific method, whilst in other fields it might be much harder to perfectly identify \textendash\ perhaps it refers to the number of sources required to create meaningful historiography or the lenses through which we may decide whether a piece of art is ``good''. This difficulty to pinpoint and thus define its value as a corrective, as well as its correlation to dissemination of knowledge is what will be qualified throughout this essay.

The need for empiricism and reasoning in the modern sciences allows for a robust framework through which errors can be found, including when faced with possible paradigm shifts. Before the formalisation of the scientific method, often one's reputation was the most important aspect in determining whether their observations and hypothesis were taken as a fact, such as Lamarck's theory of evolution, which is later replaced by Darwin's.. On the other hand, the modern tradition demands peer-review and reproducibility, lending itself to the spotting errors, hoaxes or otherwise bad methodology, as was the case when in 2011, a group of scientists thought to have found particles which travel faster than light, directly contradicting the scientific consensus that nothing can travel faster than light. Upon peer-review a mistake in the experiment was spotted and the whole idea was dropped. The exact same set of principles was used in 1859 by Louis Pasteur to disprove spontaneous generation \textendash\ the idea that, for instance, leaving a towel with bread crumbs in the corner of one's house would spontaneously generate rats. Pasteur's discoveries, which were based on the rigour and reproducibility required by the scientific method, clashed directly with observations that could be made by anyone (i.e., rats did exist where they once did not), and presented a logical explanation to what looked like spontaneous generation \textendash\ boiling down to the existence of foreign bodies, that is, bacteria entering a meat soup (Pasteur's experiment), or rats coming from the outside and hiding in our example towel. Furthermore, scrutinising the scientific method is in itself a large part of the tradition, which goes a long way in both correcting the field's ways of knowing (reason and the conclusions which are drawn), and the methodology (the set-up and collection of data).

There is an implied understanding that the traditions compensate for human biases and lead to an empirical science, but that is not quite the whole picture. Often we think there is very little space for emotion and/or faith in science, but on the verge of a breakthrough or under the pressure to produce meaningful results, it is hard to argue that scientists may be guilty of wishful thinking \citep{PlanetMoney677} \textendash\ that perhaps one more data point will show the correlation one is sure exists or might just allow to get the funding that is required to continue the research. Indeed, the scientific method suffers greatly from these ways of knowing often not associated with it. In 2014, Brian Nosek, editor of the journal Social Psychology, and his team attempted to replicate one hundred experiments published in respected journals of a previous year. Out of them, 64\% showed statistically irrelevant results, and two short-comings were suggested \citep{PlanetMoney677}: First, there was a lack of rigour in the methodology caused by the scientists' wishful thinking (and other external and internal motivations, such as funding, as discussed previously); and second, the focus lies on the production of knowledge, and not its replication, implying that even though the tradition requires reproducibility, which would invalidate an experiment with faulty methodology \textendash\ because this basic tenet of the tradition is not being utilised, then in practical terms, the scientific method (in this context) does not, at all, offer a corrective.

In stark contrast to the albeit faulty but well defined standard in science, the arts enjoy a framework that is far more subjective. Its traditions are used to assess both shared and personal knowledge: For example, is this painting naturalist or romantic? Is this painting ``good''? Or to in some way validate a certain individual or group \textendash\ giving out an award for best film, for example. Whether a piece belongs to a literary movement involves evaluating motif, style, technique and more. If we consider romanticism: it was established due to similarities between the likes of Lord Byron and Victor Hugo, and their distancing from the contemporary movement; that is, the post-French revolution, loathing sentiment went from personal to shared knowledge, forming a new tradition. Conversely, if we accept that establishing whether a piece is good is one of the traditions of art, then do these traditions shape our personal knowledge? What constitutes good art is ultimately up to personal judgement \textendash\ often based on aesthetics, emotional appeal, imagery evoked, etc, though inevitably influenced by one's context. If presented with the modernist Abaporu by Tarsila do Amaral and the paintings by Eug\`{e}ne Delacroix within the context of romanticism (1770s), one would be hard-pressed to affirm Amaral's work to be good, as it completely deviates from the sorrow (or nationalism, depending on the exact year and country of the movement) one would expect from romanticism; but given Abaporu in its rightful context, it is outstanding \textendash\ showing the disproportional use one's hands and feet (manual labour), and atrophy of the mind, living in the Brazilian semi-arid \textit{sert\~{a}o} (``backcountry''). This is to say that traditions heavily shape art, both in its creation, and appreciation by showing either what moulds to break, the context for which it has appeal or what style is currently being highly valued by society.

If traditions in arts, thus, are created by person or persons that act as gatekeepers, can we then conclude traditions are just an extension of their personal knowledge? The discussion presented affirms that yes, we can, and that has a myriad of implications regarding the parallels drawn between correctives offered by natural science and arts. Most importantly, the existence of a certain literary movement that appreciates a specific aesthetic cannot be used as a standard to define beauty. This would imply that a person or group's opinion trumps another's \textendash\ though we often do follow and value the opinions of celebrities, for example. The emphasis here is on the word opinion, which lies completely opposite to the scientific method, and discredits the idea of putting the traditions of both areas of knowledge on the same scale. Regarding art, the main question can then be rephrased to ``Do established opinions often supersede personal opinions?'', and the answer is a resounding yes, and this is very clearly reflected on our appreciation art, emotional responses to it, prices paid for an artist compared to another, and much more.

In conclusion, it is important to differentiate between the rigorousness of the traditions in different areas of knowledge in order to discuss the production of knowledge. It can be affirmed that they do, indeed, offer correctives, but the quality of the knowledge created depends greatly on how well established the process is. By opposing art and science, we were able to compare and contrast the subjectivity of the former with the objectivity of the latter, as well as see the process of production of knowledge \textendash\ whereas in science a model is presented to justify one's observations, the arts deal with movements that for varying reasons triumph over others.

\bibliography{bibliography}
\end{document}
