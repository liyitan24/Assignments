The experiment aims to determine the concentration of unstandardised \ce{KIO3} using a redox titration. We do so by reacting the potassium iodate with excess potassium iodide in an acidic solution. The product is then titrated with sodium thiosulfate with the aid of starch as an indicator, as per the following equations:
\begin{equation}\label{eq:KIO3toI2}
	\ce{KIO3(aq) + 5KI(aq) + 6H+ -> 3I2(aq) + 6K+(aq) + 3H2O(l)}
\end{equation}

\begin{equation}\label{eq:Indicator}
	\ce{2Na2S2O3(aq) + I2(aq) -> 2NaI(aq) + Na2S4O6(aq)}
\end{equation}

We rely on recognising colour changes in order to assess whether the reactions have taken place. For (\ref{eq:KIO3toI2}), the iodine should give the solution a yellow-brown colour, while for (\ref{eq:Indicator}) we should expect a change from dark blue to colourless when the iodine has been fully reacted.