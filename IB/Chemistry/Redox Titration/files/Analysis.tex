For 0th titration, we can find the amount of sodium thiosulfate(n) from its concentration (c) and volume (V):
\begin{equation*}
 n(\ce{Na2S2O3}) = V \cdot c = \SI{0.003040}{\mol}
\end{equation*}

From (\ref{eq:Indicator}) we can calculate the amount of \ce{I2} reacted:
\begin{equation*}
 n(\ce{I2})(reacted) = \frac{1}{2}n(\ce{Na2S2O3}) = \SI{0.001520}{\mol}
\end{equation*}

Equation (\ref{eq:KIO3toI2}) tells us that each mol of \ce{I2} came from a 1:3 ratio from \ce{KIO3}, which we can then divide by the initial volume and find our concentration:
\begin{equation*}
\begin{split}
 &n(\ce{KIO3}) = 3\cdot n(\ce{I2}) = \SI{0.004560}{\mol} \\
 &\therefore c(\ce{KIO3}) = \frac{ \SI{0.004560}{\mol} }{ \SI{5.00}{ml} } = \SI{0.9120}{\molar}
\end{split}
\end{equation*}

Finally, we may add the relative uncertainties in the volume of the titrant and volume of \ce{KIO3}:
\begin{equation*}
\frac{\SI{0.20}{ml} }{ \SI{30.40}{ml} } + \frac{ \SI{0.05}{ml} }{ \SI{5.00}{ml} } = 0.016
\end{equation*}

Repeating this process for all titrations:
\begin{table}[h] \centering
\begin{tabular}{c c}
	Index & c(\ce{KIO3}) \si{\molar}\\ \hline
	0	&	\SI{0.91\pm0.01}{\molar}\\
	1	&	\SI{0.94\pm0.02}{\molar}\\
	2	&	\SI{0.89\pm0.01}{\molar}\\
	3	&	\SI{0.89\pm0.01}{\molar}\\
\end{tabular}
\end{table}