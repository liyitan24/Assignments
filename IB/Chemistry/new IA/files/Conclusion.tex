In this experiment we attempted to experimentally determine the effect of salinity on the specific heat capacity of water. We have arrived at the following table, which does describe the general trend we would expect. This was done by utilising a kettle to boil about \SI{750}{\gram} of water with different masses of salt, and recording the change in temperature over a time period.
\begin{table}[h] \centering
	\begin{tabular}{cc} 
		Salinity & Specific heat capacity \\
		(\si{\gram\per\gram}) & (\si{\joule\per\gram\per\celsius}) \\ \hline
		0\%	& $4.18\pm 0.40$\\
		1\%	& $3.88\pm 0.40$\\
		2\%	& $3.83\pm 0.40$\\
		3\% & $3.53\pm 0.40$\\
	\end{tabular}
	\caption{Specific heat capacity of water at different salinities.}
\end{table}

From comparing to table values, we would expect a change from 4.18 to 3.99 \citep{KayeLaby2005TableOfContents}, which is a difference of about $ \frac{0.19}{4.18} = 4.5\% $. Given that the uncertainty of the kettle was approximately 8\%, it was possible to find accurate, though imprecise, values. This further cements the importance . 
\paragraph{To add:} Comparison to table value and how off our found result is. Sources of uncertainty