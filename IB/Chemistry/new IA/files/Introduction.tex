Specific heat capacity is a measure of the amount of energy required to change the temperature of a 1 kg object by 1 K. 
It is an important concept as it gives an indication of how much energy and time is required to heat or cool an object and consequently the cost \textendash\ something of significant importance for many industries. 
For simplicity we often assume that water and aqueous solutions have the same specific heat capacity. 
The fundamental reason as to why there is such a change is beyond the scope of this paper, however, attempting to quantify it presents an interesting question. 
In order to minimise any safety concern and the ease of reproducibility, I opted specifically for investigating the relationship between salinity and specific heat capacity \textendash\ from tap water up to about 3\% salinity, approximately what is found in seawater.

As we seek to explore the relationship between the salinity and the solution’s specific heat capacity, this exploration will consist of using a kettle to heat up water of varying salinities, using software to track the change in temperature, and ultimately calculate their specific heat capacity. Formally, the research question is: \textbf{How does the salinity of water affect its specific heat capacity?}