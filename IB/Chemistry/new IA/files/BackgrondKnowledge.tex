We may express a relationship between heat, a form of energy, and the following characteristics of a material: mass ($m$), specific heat capacity ($c$) and the change in the temperature it undergoes ($\Delta T$), as follows:
\begin{equation*}
	\Delta H = mc \Delta  T
\end{equation*}

Central to our calculation is the concept of power, which we may tie with the heat equation:
\begin{equation}\label{eq:Power}
\begin{split}
	Power &= \frac{Energy}{time} \\
	\therefore P &= \frac{mc\Delta T}{\Delta t}
\end{split}
\end{equation}

Finally, when dealing with a change in phase (such as from liquid to gas), it is also important to discuss the amount of energy required to undergo that change at a constant temperature; hence, a plateau will be seen at the end of the time x temperature graphs that later will be presented. As this does not concern heat capacity, it will be trimmed from the graphs.