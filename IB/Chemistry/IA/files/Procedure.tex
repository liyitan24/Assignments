\subsection{Acidified Potassium Dichromate}
In order to reduce the relative uncertainty, 1 litre of the solution was prepared, though a far smaller amount of it was used.
\begin{enumerate}
	\item In a 1000.00 ml $\pm$ 0.40 mL volumetric flask, 33.790 g $\pm$ 0.001 g of Potassium dichromate ($\ce{K2Cr2O7}$) were added onto approximately 500ml of distilled water.
	
	\item 325 ml of concentrated sulphuric acid were slowly added onto the volumetric flask. The solution very quickly became too hot to touch. Absolute precision is not necessary at this step, as the sulphuric acid only serves to provide $H+$ to the reaction, which will be in excess.
	
	\item The solution was allowed to cool down to room temperature, and the volumetric flask filled to its 1000 ml mark.
\end{enumerate}

Thus we may calculate the concentration of the solution based on the molar mass of $\SI{294.18}{g\per\mole}$:
\begin{equation}\begin{split}
&c(\ce{K2CR2O7}) = \frac{n}{V} = \frac{m}{VM} =
\frac{ (33.790\pm 0.001)\si{g}}{ (294.18 \pm 0.01)\si{g\per\mol} \cdot (1000.00 \pm 0.40)\si{\milli\litre}} \\
&\therefore c(\ce{K2CR2O7}) = (1.1486 \pm 0.0005 )\cdot 10^{-4} \si{\mol\per\milli\litre}
\end{split}
\end{equation}

The above calculation of the uncertainty in concentration ($\Delta c$) is done as follows:
\begin{equation*}
\begin{split}
	& \Delta c(\ce{K2CR2O7}) = c \cdot \left ( \frac{0.001}{33.790} + \frac{0.01}{294.18} + \frac{0.40}{1000.00} \right ) \\
	\therefore & \Delta c(\ce{K2CR2O7}) = 0.0005
\end{split}
\end{equation*} 

\subsection{Acidified Ferrous Ammonium Sulphate}
The ferrous ammonium sulphate hexahydrate, $\ce{Fe(NH4)2(SO4)2 \cdot {6} H2O}$ (FAS), for the back-titration was prepared as follows:
\begin{enumerate}
	\item In a 250.00 ml $\pm$ 0.23 ml volumetric flask, 36.450 g $\pm$ 0.001 g of FAS was added to about 190 ml of distilled water.
	
	\item 6.3 ml of concentrated sulphuric acid, $\ce{H2SO4}$, were added into the flask. (Same considerations regarding precision as the other solution)
	
	\item The volumetric flask is filled with water to its 250 ml mark.
\end{enumerate}

Thus we may calculate the concentration based on the molar mass of $392.13 \si{g\per\mole}$:

\begin{equation}
\begin{split}
&c(FAS) = \frac{ (36.450 \pm 0.001) \si{g} }{ (392.13 \pm 0.01) \si{g\per\mole} \cdot (250.00 \pm 0.23)\si{ml} }\\
&\therefore c(FAS) = (3.7181 \pm 0.0036) \cdot 10^{-4} \si{\mole\per\ml}
\end{split}
\end{equation}

\subsection{Preparation of Samples}
For the distilled sample, $20.0 \pm \SI{0.3}{\milli\litre}$ were added to approximately 60 ml of water and distilled until about 40 ml of the ethanol + water solution were accumulated. The solution was moved to a 100.0 ml volumetric flask and filled with distilled water.

For the undistilled sample: In a 100.0 ml volumetric flask, 20.0 ml of wine were added, and then filled with distilled water.

In both cases, the same bottle of wine is used with 12\% ABV(alcohol by volume).

\subsection{Experiment}

\begin{itemize}
	\item In a 250 ml flask in an ice bath, $\SI{40.0}{\milli\litre} \pm \SI{0.1}{\milli\litre}$ of the sample was pipetted, and then slowly $\SI{150.0}{\milli\litre} \pm 0.1 \si{\milli\litre}$ of the dichromate solution were added. The ice bath is meant to prevent ethanal from escaping as a gas, however almost immediately there was a sweet smell in the air, indicating there had been some ethanal escaping. This was done for both samples.
	\item A cork was put on the flasks and both were placed in a water bath at $\SI{60}{\celsius}$ for 30 minutes, then allowed to cool.
	\item A burette was filled with FAS and $\SI{5.00}{\milli\litre} \pm \SI{0.05}{\milli\litre}$ of the sample were prepared for titration with 5 drops of 1,10-Phenanthroline Ferrous Sulphate indicator.
\end{itemize}

I suspect the largest source of error in the experiment will come from the aldehyde not fully oxiding, which would be noticeable in two different stages: 1) When the dichromate was poured onto the alcohol. As mentioned previously, the ice bath is used so that the ethanal remains in liquid state; and 2) after the thermostatic water bath, as the solution will be at $\SI{60}{\celsius}$, any ethanal left will immediately escape when the cork is removed from the beaker.