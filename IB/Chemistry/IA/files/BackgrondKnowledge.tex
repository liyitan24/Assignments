The focus of the procedure is to measure the amount of oxidising agent (potassium dichromate) left in the solution after all of the ethanol has been converted to ethanoic acid. We are able to quantify this through a back-titration utilising ferrous ammonium sulphate as titrant. As will be seen below, both reactions require a large amount of available $\ce{H+}$, hence both solutions are acidified with concentrated sulphuric acid \citep{FergSirromet}.

The two half-steps of the oxidation are as follows:
\begin{equation}\label{eq:EthanolToAldehyde}
\ce{3CH3CH2OH + Cr2O7^{2-} + 8H+ -> 3CH3CHO + 2 Cr^{3+} + 7H2O}
\end{equation}
\begin{equation}\label{eq:AldehydeToAcid}
\ce{3CH3CHO + Cr2O7^{2-} + 8H+ -> 3CH3COOH + 4H2O + 2 Cr^{3+}}
\end{equation}

which result in the following complete equation:

\begin{equation} \label{eq:EthanolToAcid}
\ce{2Cr2O7^{2-} + 3 CH3CH2OH + 16 H+ -> 4 Cr^{3+} + 3 CH3COOH + 11 H2O}
\end{equation}

And the back-titration is as follows:
\begin{equation} \label{eq:Titration}
\ce{ Cr2O7^{2-} + 6 Fe^{2+} + 14H+ -> 2Cr^{3+} + 6Fe^{3+} + 7 H2O }
\end{equation}

Finally, throughout the calculations we will be using the standard letters to represent concepts followed by a subscript, such as concentration: c; amount: n; molar mass: M; volume: V; mass: m; and so m($\ce{H2O}$) would represent the mass of $\ce{H2O}$.