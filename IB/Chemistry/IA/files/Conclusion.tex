This paper aimed to estimate the alcohol content of two samples of a bottle of wine with 12\% ABV: one distilled and another undistilled. The values found were $(10.8 \pm 1.0)\%$ and $(15.6\pm 2.0)\%$ respectively \textendash\ reasonably close to what I expected, considering the qualitative uncertainties in the process.

We speculated that a couple of situations in which ethanal could leak and consequently increase our uncertainty, and this would be noticed by a sweet, fruity smell. The smell was noticed both when dichromate was added to the alcohol and upon leaving the water bath. Furthermore, assessing the colour of the solution during titration presented a larger challenge than expected. This random uncertainty is reflected on our raw data, through the variance in the titrants, and is responsible for about half of our approximately 10\% uncertainty. This was in part tackled during the experiment by changing to a beaker with smaller diameter and by dilluting the titrate, however neither of them were good enough. A possible improvement would be to combine the methods used with a halogen lamp \citep{FergSirromet}.

The purpose of performing this experiment with undistilled wine was to find out how much impurity we could expect when utilising the same method as a homebrewer. Obviously we can't affirm that all homemade wine or beer will have about 5\% of other oxidisible compounds, but this is a really good starting point as to what to expect. Reflecting after the experiment, the same results could have more reliably been achieved by oxidising what was left-over from the distillation, as we would be dealing with far less volatile compounds than ethanal.