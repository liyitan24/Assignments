The idea for this investigation came from my personal need to establish the alcohol content of home-brewed beer. Home-brewers often utilise a hydrometer to compare the specific gravity before and after fermentation is complete \citep{Flowers2014Keg}, but I forgot to do the former, which meant I had to establish the alcohol percentage based on the finished product alone. This need led me to my research question: \textbf{How can we determine the alcohol content of wine?}

Both wine and beer are produced through a similar process, and although I would speculate that the percentage of other compounds that may be oxidised (such as sulphites, tannins and sugar) is different, by distilling the wine we are able to separate alcohol and water from the other compounds \textendash\ allowing us to deal with a possibly large source of error and any large differences in other compounds that may exist in the different alcoholic beverages.

The procedure is an extension of an experiment done in the classroom, which consisted of oxidising propanols in order to determine whether it was a primary or secondary alcohol. It became clear a similar experiment could be done with the alcohol in wine, ethanol \textendash\ The distinction being that ethanol can only be a primary alcohol \citep[pp.491-493]{ChemIB2014Pearson}. In short, the experiment consists of the alcohol's oxidation to aldehyde, which in turn is oxidised to a carboxylic acid, and finally back-titrated \textendash\ thus determining the amount of alcohol that was oxidised.

Before describing the experiment in detail, a few considerations are really important regarding the motivation and safety of this experiment: Home distillation is illegal in many countries, thus the experiment will be conducted with both distilled and undistilled wine, as it may provide insight into what errors can be expected when replicating as a home-brewer. Furthermore, we will be dealing with strong oxidants and acids, so proper handling and safety equipment (gloves and glasses) is of the utmost importance. Although this may be easily replicable in one's home-brewery space, I cannot emphasise enough how much easier and safer it is to utilise the standard, a hydrometer.
