Conservation of energy is central to our experiment. By assuming that the system is perfectly isolated, we say that no energy is being gained or lost by the system, which can be expressed by the following equation:
\begin{equation}
	\sum \Delta H = 0
\end{equation}
where $\Delta H$ refers to heat, the amount of energy that is transferred from a hotter to a colder body due to their difference in temperature. This means that we can affirm that all energy lost by the warm cheese has to be gained by the water and the brass cup; that is:
\begin{equation} \label{eq:ConservationOfHeat}
	\Delta H[cheese] = -(\Delta H[brass] + \Delta H[water])
\end{equation}

Furthermore, we may express a relationship between heat and the following characteristics of a material: mass (m), specific heat capacity (c) and the change in the temperature it undergoes ($\Delta T$), as follows:
\begin{equation} \label{eq:DeltaH}
	\Delta H = mc\Delta T
\end{equation}

When dealing with a change in phase (such as from solid to liquid), it is also important to discuss the amount of energy required in order to undergo that change, however cheese seems to be amorphous and changing viscosity rather than changing phase (Miroshnychenko, Y., Personal communication, January, 2018). Therefore, we are able to completely disregard latent heat from our calculations, though a short consideration of its effect will be done in the evaluation.

Finally, Newton's law of cooling will be important when considering the sources of error and improvements to this experiment. In short, it states that the rate of heat loss is proportional to the difference in temperature between the bodies. It is easily understood by the following equation, though the mathematics of it will not be needed for our purposes \citep[pp.12-18]{Lienhard2001Heat}:
\begin{equation}
	\frac{\partial T}{\partial t} \propto \frac{\partial {^2} T}{\partial x^2}
\end{equation}
where T is a function of temperature that depends on space (x) and time (t) and k is just a constant of proportionality.
