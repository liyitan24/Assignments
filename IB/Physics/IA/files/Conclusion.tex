In this experiment we were able to experimentally determine the specific heat capacity of a particular type of cheese to be $\SI{2.32}{\joule\per\celsius\per\gram}$. This was done by melting a known mass of cheese, pouring it into a cup of water, and measuring the water's temperature increase. Utilising the idea of conservation of energy we expected all of the heat lost by the cheese to move to the water, but this was not necessarily the case as some is lost to the environment.

The uncertainty associated with the instruments was largely made irrelevant by the relative size of the experimental uncertainty (15\% vs 2\%). As such, there is a fairly low degree of certainty in the results found.

Cheese was assumed to be amorphous, and while I'm still confident in this assumption, it might be interesting to briefly discuss if the supposition is wrong. With energy being released from liquid to solid, we would expect a higher final temperature, and consequently we would find a smaller specific heat capacity.