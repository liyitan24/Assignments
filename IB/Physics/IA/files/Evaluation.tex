The uncertainty in this experiment is overwhelmingly experimental, approximately 15\%, which mostly reflects the difficulty in assessing the thermal equilibrium. Perhaps the biggest change that can be done is improving the insulation of the brass cup. This means that irrespective of the reduced cooling rate (as per Newton's law of cooling), as no heat is lost to the environment, we are able to wait a long time for thermal equilibrium. As it stands, we are likely to perceive a smaller change in the water's temperature, similar to the data collected before the ultimate experimental procedure – a comparison of which can be seen below.

\begin{table}[h] \centering
  \begin{tabular} {c|c|c|c}
  	Index & $T_0$(Cheese) $\pm 0.1 \si{\celsius}$& $T_0$(Water) $\pm 0.1 \si{\celsius}$& $T_f$(Water) $ \pm 0.1 \si{\celsius}$\\ \hline
  	Outlier & 59.8 &	27.5 &	31.8 \\
    1 & 55.6 &	20.1 &	30.3 \\
  \end{tabular}
\end{table}

Handling cheese as it changes viscosity also posed various difficulties, from arriving at a procedure that works, to the best way to handle the beakers, and how to make accurate measurements. The very first attempt at a procedure was similar to how the experiment is done for solids, utilising a clamp to hold the beaker containing cheese inside a kettle. Given enough time it was likely that it would work, but between the heating clamp and the agitation of the water inside the kettle, the risk of getting burnt was far too high.

The procedure that was settled on showed itself to be much safer, as the beakers were placed in a pot with warm water (as opposed to boiling). This had the added benefit of allowing the cheese inside the beakers to be stirred in order to achieve a uniform temperature. While it was a clear improvement to the previous procedure, it was still difficult to verify that the cheese had a uniform temperature, though we can be far more certain.

As the cheese was placed into the water, the outside would cool immediately, slowing down the conduction of heat to the inside, as per Newton's law of cooling. The solution was to swirl the cheese inside the water, which we expected to improve the cooling by conduction. Perhaps the cheese could be separated into smaller portions as it is poured into the liquid. It is somewhat impractical to do consistently (that is, all smaller pieces to have the same size), but it might be better than more repetitions of a smaller mass of cheese \textendash\ which would result in a larger relative uncertainty.
