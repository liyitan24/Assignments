The equipment necessary is as follows:
\begin{itemize}
	\item A water bath, or a pot filled with water to approximately $\SI{60}{\celsius}$.
	\item pieces of cheese of approximately the same mass
	\item beakers
	\item 2 thermometers. One utilised on the cheese for stirring and verifying the uniformity of the temperature, and the other for the water cup.
	\item a thermally insulated brass cup with lid, filled with a known mass of water. This consisted of a brass cup placed inside of a thick styrofoam cup which in turn was inside another brass cup. The brass cup is chosen due to having a known specific heat capacity, but because brass is a good heat conductor, we surround it with styrofoam \textendash\ a poor heat conductor \textendash\ in order to ``isolate'' the system. The outermost cup is simply for transport.
\end{itemize}

The experimental procedure is the following:
\begin{enumerate}
	\item Measure the mass of the inner brass cup alone (without the Styrofoam cup): $\SI{187.7}{g} \pm \SI{0.1}{g}$
	\item $\SI{47.3}{g} \pm \SI{0.1}{g}$ of water is placed inside the brass cup, and its temperature is recorded.
	\item $\SI{38.49}{g} \pm \SI{0.20}{g}$ of cheese is placed inside a beaker, which is in turn placed in the water bath until reaching approximately $\SI{60}{\celsius}$.
	\item The cheese is then placed in the brass cup with water and the maximum temperature of the water is recorded.
\end{enumerate}

\textbf{Note that there is no particular reason for the control variables of 38.5 g of cheese or 47.3 g of water, they are simply the first measurements taken, which were replicated throughout the experiment.} From previous attempts, it seems that the masses just have to be in the same order of magnitude.

\subsection{Variables}
\begin{itemize}
	\item Independent: Initial temperature of cheese
	\item Control: mass of water and cheese, described in detail in Table \ref{tb:ControlVariables}
\end{itemize}
\begin{table} [h]
	\centering
	\begin{tabular} {c|c|c|c|c}
		\textbf{c(brass)} & \textbf{m(brass)} & \textbf{c(water)} & \textbf{m(water)} & \textbf{m(cheese)} \\
		$\si{\joule\per\gram\per\kelvin}$ & $\si{g} \pm \SI{0.01}{g}$ & $\si{\joule\per\gram\per\kelvin}$ & $ \si{g} \pm \SI{0.1}{g}$ & g $\pm \SI{0.2}{g}$\\ \hline
		0.380 & 187.74 & 4.186 & 47.3 & 38.5
	\end{tabular}
	\caption{Control variables for the experiment. c denotes specific heat capacity and m denotes mass. Note that where not specified, value is taken from course book \citep{IBPhysics} and uncertainty taken to be zero.}
	\label{tb:ControlVariables}
\end{table}

While the the mass of water and brass cup have instrumental uncertainty, the cheese was cut by hand to approximately 38.5 g, and so the value reflects taking the mean mass from all the pieces of cheese and the respective uncertainty.
