\subsection{Data} \label{Data}
\subsubsection{Primes up to $10^8$} \label{data:PrimesUpTo10^8}
\begin{table}[h]
	\begin{tabular}{c|l}
		$n$   & Number of Primes less than $n$ \\ \hline
		$10 $ & 4\\
		$10^{2} $ & 25\\
		$10^{3} $ & 168\\
		$10^{4} $ & 1229\\
		$10^{5} $ & 9592\\
		$10^{6} $ & 78498\\
		$10^{7} $ & 664579\\
		$10^{8} $ & 5761455\\
		
		
	\end{tabular}
\end{table}
\subsubsection{$Li(100^{\rho}) + Li(100^{\bar{\rho}})$}\label{tb:LiValuesFor100}
\begin{table} [h] \centering
	\pgfplotstabletypeset[
	every head row/.style={output empty row},
	/pgf/number format/fixed zerofill={true}, /pgf/number format/precision=7, 
	]
	{li100.dat}
	
	\caption{$Li(100^{\rho}) + Li(100^{\bar{\rho}})$ for the first forty $\rho$. Algorithm: Appendix \ref{ap:MathematicaLi}}
\end{table}


\vspace{5cm}
\subsubsection{$Ei(100^{\rho}) + Ei(100^{\bar{\rho}})$}\label{tb:EiValuesFor100}
\begin{table}[h] \centering
	\pgfplotstabletypeset[
	every head row/.style={output empty row},
	/pgf/number format/fixed zerofill={true}, /pgf/number format/precision=7, 
	]
	{ei100.dat}
	\caption{First forty values for $Ei(100^{\rho}) + Ei(100^{\bar{\rho}})$ (Algorithm: Appendix \ref{ap:MathematicaEi})}
\end{table}

\subsection{Algorithms}
\subsubsection{Sieve of Eratosthenes in Typescript} \label{ap:Sieve}

\begin{verbatim}
function sieveOfErathosthenes(max: number): Array<number> {
 let flags: Array<boolean> = [];
 let primes: Array<number> = [];
 let prime = 2;

 let n: number = max;
 while(n--) { flags[max-n] = true;}

 for (prime = 2; prime < Math.sqrt(max); prime++) {
  if (flags[prime]) {
   for (let j = prime + prime; j < max; j += prime) { flags[j] = false;}
  }}

 for (let i = 2; i < max; i++) {
  if (flags[i]) {primes.push(i);}
 }
 return primes;
}
\end{verbatim}

\subsubsection{Mathematica}
The variable \verb|data| refers to the imaginary portion of the zeros\citep{UMNZetaZeros}.
\paragraph{Li} \label{ap:MathematicaLi}
\begin{verbatim}
f1 = 100.0^(0.5 + Part[data]I)
f2 = 100.0^(0.5 - Part[data]I)
TableForm[LogIntegral[f1] + LogIntegral[f2] - 2LogIntegral[2]]
\end{verbatim}
\paragraph{Ei} \label{ap:MathematicaEi}
\begin{verbatim}
f1 = (0.5 + Part[data]I)*Log[100.0]
f2 = (0.5 - Part[data]I)*Log[100.0]
TableForm[ExpIntegralEi[f1] + ExpIntegralEi[f2]]
\end{verbatim}
\paragraph{Sum of forty zeros of $Ei(\ln 100^{\rho})$ to $Ei(\ln 100^\frac{\rho}{6})$.} \label{ap:MathematicaSumOfEi}
\begin{verbatim}
f1a = (0.5 + Part[data]I)*Log[100.0^(1/1)]
f1b = (0.5 - Part[data]I)*Log[100.0^(1/1)]
f2a = (0.5 + Part[data]I)*Log[100.0^(1/2)]
f2b = (0.5 - Part[data]I)*Log[100.0^(1/2)]
f3a = (0.5 + Part[data]I)*Log[100.0^(1/3)]
f3b = (0.5 - Part[data]I)*Log[100.0^(1/3)]
f5a = (0.5 + Part[data]I)*Log[100.0^(1/5)]
f5b = (0.5 - Part[data]I)*Log[100.0^(1/5)]
f6a = (0.5 + Part[data]I)*Log[100.0^(1/6)]
f6b = (0.5 - Part[data]I)*Log[100.0^(1/6)]

TableForm[(1/1)(ExpIntegralEi[f1a] + ExpIntegralEi[f1b])]
TableForm[(-1/2)(ExpIntegralEi[f2a] + ExpIntegralEi[f2b])]
TableForm[(-1/3)(ExpIntegralEi[f3a] + ExpIntegralEi[f3b])]
TableForm[(-1/5)(ExpIntegralEi[f5a] + ExpIntegralEi[f5b])]
TableForm[(1/6)(ExpIntegralEi[f6a] + ExpIntegralEi[f6b])]
\end{verbatim}


\subsection{Equations} \label{Equations}
\subsubsection{Complex plain zeta function}
\begin{equation} \label{eq:Appendix-RiemannModifiedSum}
\zeta(s) := (1-2^{1-s})^{-1} \sum_{n=1}^{\infty} \frac{(-1)^{n+1}}{n^s}, \ 0 < \Re(s) < 1
\end{equation}

\begin{equation} \label{eq:Appendix-RiemannFunctional}
\begin{split}
\Gamma(s) = \int_0^\infty x^{s-1} e^{-x}\, dx \\
\zeta(s) := 2^s\pi^{s-1}\ \sin\left(\frac{\pi s}{2}\right)\ \Gamma(1-s)\ \zeta(1-s), \ \Re(s) < 0
\end{split}
\end{equation}

\subsubsection{Weighted prime counting function}
\begin{equation}\label{eq:Appendix-Jn}
\begin{split}
	J(n) = \sum_{k=1}^{\epsilon} \frac{1}{k}\pi(\sqrt[k]{n}), \sqrt[k]{\epsilon} \geq 2 \text{ and } n \in \mathbb{R^+} \\
	\pi(n) = 0 \text{ for $n < 2$, hence the stipulation of} \sqrt[k]{\epsilon} \geq 2
\end{split}
\end{equation}

\begin{equation}\begin{split}
	J(100) &= \pi(100) + \frac{1}{2}\pi(\sqrt[2]{100}) + \frac{1}{3}\pi(\sqrt[3]{100}) + \frac{1}{4}\pi(\sqrt[4]{100})+ \frac{1}{5}\pi(\sqrt[5]{100}) + \frac{1}{6}\pi(\sqrt[6]{100}) \\
	&=25 + \frac{1}{2}\pi(10) + \frac{1}{3}\pi(4.642) + \frac{1}{4}\pi(3.162) + \frac{1}{5}\pi(2.512) + \frac{1}{6}\pi(2.154) \\
	&=25 + \frac{1}{2}4 + \frac{1}{3}2 + \frac{1}{4}2 + \frac{1}{5} + \frac{1}{6} = 28.53\bar{3}
\end{split}
\end{equation}

