Let us focus on $B_n$
\begin{equation*}
B_n = \frac{2}{L} \int_0^L (\psi(x,0)) \sin \left( \frac{n\pi x}{L} \right) dx \quad \quad n = 1,2,3,\ldots \\
\end{equation*}

This seems like it could be solved with integration by parts, but we must first understand the function $\psi(x,0)$. 
We do so by graphing with the help of $u_e(x)$ and $u(x,0)$ through the expression (\ref{eq:u-psi-ue}) in its domain of $0 \le x \le L$. 
Of note is that, since $u(x,0)$ is a step function, we can look at it as a graph transformation, specifically a vertical shift, aiding in finding the graph for $\psi(x,0)$. This can be seen in figure \ref{gr:Integrals-of-u-ue-psi}.

\begin{figure}[h] \centering
\begin{tikzpicture}[domain=0:4, scale=0.7]
    \draw[very thin,color=gray,opacity=0.2] (-0.1,-1.1) grid (3.9,3.9);
    \draw[->,opacity=0.2] (-0.2,0) -- (4.2,0) node[right,opacity=1] {$x$}; 
    \draw[->,opacity=0.2] (0,-1.2) -- (0,4.2) node[above,opacity=1] {$u(x,0)$};
    \draw (0,3.1) node [anchor=east] {$T_{\max}$}; \draw (0,1.1) node [anchor=east] {$T_{\min}$};
    \draw (3.5,-0.2) node  {L}; \draw[dotted] (3.5,0) -- (3.5,1.0);
    
    \draw[thick] (0.1,1.0) -- (3.5,1.0); \draw [thick] (0,1.0) circle (0.1); \fill[black] (0,3) circle (0.1);
    \draw[pattern=north west lines, pattern color=blue, opacity = 0.2] (0.01,0) rectangle (3.49,0.97);
\end{tikzpicture}
\begin{tikzpicture}[domain=0:4, scale=0.7]
    \draw[very thin,color=gray,opacity=0.2] (-0.1,-1.1) grid (3.9,3.9);
    \draw[->, opacity=0.2] (-0.2,0) -- (4.2,0) node[right,opacity=1] {$x$}; 
    \draw[->, opacity=0.2] (0,-1.2) -- (0,4.2) node[above,opacity=1] {$u_e(x)$};
    \draw (0,3.1) node [anchor=east] {$T_{\max}$}; \draw (0,1.1) node [anchor=east] {$T_{\min}$};
    \draw (3.5,-0.2) node  {L}; \draw[dotted] (3.5,0) -- (3.5,1.0);

    \draw[thick] (0,3.0) -- (3.5,1.0);
    \draw[pattern=north west lines, pattern color=blue, opacity = 0.2] (0.01,0) -- (0.01,3.0)-- (3.49,0.97) -- (3.49,0) -- cycle;
\end{tikzpicture}
\begin{tikzpicture}[domain=0:4, scale=0.7]
    \draw[very thin,color=gray,opacity=0.2] (-0.1,-1.1) grid (3.9,3.9);
    \draw[->, opacity=0.2] (-0.2,0) -- (4.2,0) node[right,opacity=1] {$x$}; 
    \draw[->, opacity=0.2] (0,-1.2) -- (0,4.2) node[above,opacity=1] {$\psi(x,0)$};
    \draw (0,1.1) node [anchor=east] {$T_{\min}$}; \draw (0,3.1) node [anchor=east] {$T_{\max}$};
    \draw (0,2.-2.9) node [anchor=east, scale=0.8] {$(T_{\min}-T_{\max})$}; 
    \draw (3.6,0.3) node  {L};

    \draw[thick] (0.1,-1) --(3.5,0); \draw [thick] (0,-1.0) circle (0.1); \fill[black] (0,0) circle (0.1);
    \draw[pattern=north west lines, pattern color=blue, opacity = 0.2] (0.01,-1) -- (0.01,0) -- (3.5,0) -- cycle;
\end{tikzpicture}

\caption{Graphical representation of $u(x,0)$ and $u_e(x)$, from which we able to deduce the shape of $\psi(x,0)$, and the three functions' respective integrals shaded in blue}
\label{gr:Integrals-of-u-ue-psi}
\end{figure}

Thus we can solve the integrals analytically and verify our results graphically from figure \ref{gr:Integrals-of-u-ue-psi}, though we disregard the one point which is a step (at $x=0$). 
This is done for simplicity sake.
\begin{equation*}
\begin{split}
    &\int \psi(x,0)dx = \int u(x,0) dx - \int u_e(x) dx \\
    \implies &\int \psi(x,0)dx = T_{\min}x - \left[ xT_{\max} + \frac{ T_{\min} -T_{\max} }{2L}x^2 \right] \\
    \therefore &\int \psi(x,0)dx = x^2 \left( \frac{ T_{\max}-T_{\min} }{2L} \right) -x (T_{\max}-T_{\min})
\end{split}
\end{equation*}
Then $B_n$ can be integrated by parts, the step-by-step can be seen in the Appendix. 
The key to it being the recognition that the expressions with $\sin$ would always equate to zero
\begin{equation*}
\begin{split}
    B_n &= \frac{2}{L} \left( \left[ \sin\left( \frac{n\pi x}{L}\right) \int \psi(x,0)dx \right]^L_0 - \frac{n\pi}{L}\int_0^L \left[ \cos\left( \frac{n\pi x}{L}\right) \int \psi(x,0)dx \right] dx \right) \\
    &= \frac{	2( T_{\min} - T_{\max} )	}{	n\pi	}
\end{split}
\end{equation*}

Finally, we can start solving equation (\ref{eq:u-solution}) for our specified conditions, which we defined would answer our question:
\begin{equation*}
\begin{split}
T_{melt}&= u(\frac{L}{2}, t_{\max})
\\
T_{melt} &=
	T_{\max} + \frac{L}{2} \frac{ T_{\min}-T_{\max} }{ L }
	+
	\sum_{n=1}^{\infty}
	\frac{2(T_{\min}-T_{\max})}{n\pi}
	\sin \left(
		\frac{n\pi L}{2L}
	\right)
	e^{-k \left(
		\frac{n\pi}{L}
	\right)^2 t_{\max}}
\\
T_{melt} - T_{\max}&=
	\frac{L}{2} \frac{ T_{\min}-T_{\max} }{ L }
	+
	\frac{2(T_{\min}-T_{\max})}{\pi}
	\sum_{n=1}^{\infty}
		\frac{1}{n}
		\sin \left (\frac{n\pi}{2} \right)
		e^{
			-k \left(\frac{n\pi}{L} \right)^2 t_{\max}
		}
\\
\frac{T_{melt} - T_{\max}}{T_{\min}-T_{\max}}&=
	\frac{1}{2} + \frac{2}{\pi}
	\sum_{n=1}^{\infty}
	\frac{1}{n}
	\sin \left (\frac{n\pi}{2} \right)
	e^{
		-k \left(\frac{n\pi}{L} \right)^2 t_{\max}
	}
\\
\end{split}
\end{equation*}

For every \textbf{even} n, the sine function $\sin \left( \frac{n\pi}{2} \right) = 0$, and consequently so will that term of the infinite sum. For $ n = n_1 = 1 + 4k, k \in \mathbb{N} $ we will have all sine functions in the sum with the value of $1$, while for $n = n_3 = 3 + 4k$ it will equal to $-1$. 
So in order to understand the sum, we will analyse its behaviour for one element at a time and analyse their individual weights to the overall sum.pet 

First let us look at the expression given $n_1$:

\begin{equation*}
\begin{split}
&\frac{T_{melt} - T_{\max}}{T_{\min}-T_{\max}} - \frac{1}{2} =
	\frac{2}{n_1\pi}
	\cancelto{1}{
		\sin \left (\frac{n_1\pi}{2} \right)
	}
	e^{	-k \left(\frac{n_1\pi}{L} \right)^2 t_{\max}	}
\\
&\frac{2T_{melt} - T_{\max} - T_{\min}}{2(T_{\min}-T_{\max})} \frac{n_1\pi}{2}
	=
	 e^{	-k \left(\frac{n_1\pi}{L} \right)^2 t_{\max}	}
\\
&\ln \left(
	\frac{2T_{melt} - T_{\max} - T_{\min}}{2(T_{\min}-T_{\max})} \frac{n_1\pi}{2}
\right)
=
-k \left(\frac{n_1\pi}{L} \right)^2 t_{\max}
\\
&\therefore t_{max} =
	\ln \left(
		\frac{2T_{melt} - T_{\max} - T_{\min}}{2(T_{\min}-T_{\max})} \frac{n_1\pi}{2}
	\right)
	\left(
		\frac{L}{n_1\pi}
	\right)^2
	\frac{1}{k}
\end{split}
\end{equation*}

Correspondingly for $n_3$:
\begin{equation*}
    t_{n3} = \left(\frac{L}{kn_3\pi}\right)^2 \ln \left( \frac{(2T_{melt} - [T_{\max} +T_{\min}])n_3\pi }{(T_{\min} - T_{\max})(n_3\pi - 2L)} \right)
\end{equation*}

Thus, by adding the infinitely many odd $n$s that compose $t_{n3}$ and $t_{n1}$ \textendash\ each representing a fraction of the total time \textendash\, we have found a reasonably simple expression that could solve our proposed problem. 
It seems like the larger values of $n$ have a smaller ``weight'' to the overall value of the function, which we can verify ignoring the non $n$ terms for both subsets of $n, n_1$ and $n_3$:
\begin{equation*}
\begin{split}
    t_{n} \propto \frac{1}{n^2} \\
    \therefore t_{\max} \propto \sum_{n=1}^{\infty} \frac{1}{n^2}
\end{split}
\end{equation*}

From the p-series test, we know that this is a convergent sum, which means we do not have to wait infinitely for the sandwich to be ready. While it does not confirm the expression is correct, we have not been led to an absurdity. The relative weight of each term can be visualised in figure \ref{gr:1/n2}.

\begin{figure}[h] \centering
\begin{tikzpicture}
     \begin{axis}[xlabel=$n$,ylabel=$t_{n}$, axis lines=left]
        \addplot[domain=1:10,samples=101]{1/x^2};
    \end{axis}
\end{tikzpicture}
\caption{Demonstration of ${\displaystyle t_n=\frac{1}{n^2} }$, showing how quickly the weight of each $t_{n}$ decreases as $n$ increases.}
\label{gr:1/n2}
\end{figure}
It was originally stated that the acceptable time would have to be less than 10 minutes. 
This means that $t_{3}$, the second largest fraction of the total time, would contribute about 11\% or a little over a minute. 
By $n=9$, $t_{9}$'s contribution would be a little over seven seconds! 
This means that we can easily be satisfied by having only the first term of the infinite sum. 
As such, the time it takes to make a sandwich is, approximately,
\begin{equation}
    2t_{\max} = 2\left(\frac{L}{k\pi}\right)^2 \ln \left( \frac{(2T_{melt} - [T_{\max} +T_{\min}])\pi }{(T_{\min} - T_{\max})(\pi + 2L)} \right)
\end{equation}

