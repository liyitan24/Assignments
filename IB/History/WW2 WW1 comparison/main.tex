\documentclass[12pt,a4paper]{article}
\usepackage[letterpaper]{geometry}
\geometry{top=1.0in, bottom=1.0in, left=1.0in, right=1.0in}
\usepackage{setspace} \doublespacing
\usepackage{titlesec}
\newcommand{\sectionbreak}{\clearpage}

\usepackage{algorithm2e}
\usepackage[round, comma, authoryear]{natbib}
\bibliographystyle{unsrtnat}
\usepackage{import}

\begin{document}

\section*{Evaluate the importance of technology and strategy in the outcome of two 20th century wars.}

The western front of the first world war was the theatre to outstanding technological innovation, but the keyword often used to describe it is ``stalemate''. There are many technologies and tactics that contributed to this stalemate, and amongst them there are simple technologies, such as the use of barbed wire, but also the forefront of technological development, such as poison gas and tactical air support. The second world war saw even more technological advances, such as radio communication, sonar, code-breaking, the use of aeroplanes and the shear destruction of firebombs and the atomic bomb \textendash\ though this paper will focus on the warfare related technology. 

Barbed wire heavily influenced the way the first world war was fought and how an army be could prevented to cross no man's land. By cutting the flesh of infantry or cavalry who dared run across it, it provided a new layer of defence as well as a way to channel enemy soldiers into corridors, from which they could be massacred by machine guns \textendash\ another innovation of the war. The use of barbed wire was consistently effective throughout the entire war, but when considering the outcome of the war, they, and all other warfare-related technologies developed, dwarf in comparison to the importance of logistical developments.

The Great War is often described as a war of attrition, which goes a long way in explaining the importance of logistics. During the war, many of the technologies developed by one side were often countered or copied by the other side, such as the use of poison gas during the second battle of Ypres in 1915 and the subsequent development of gas masks, leading to the large stalemates seen during the war. With almost static trenches, a railway system could be used to supply the armies, greatly improving the associated speed and cost. This proved very true for the Allied forces, but not the German, whose railways were quickly overwhelmed during advances \citep[pp.138-141]{Creveld2004Supplying}, straining the resources available to the troops. During the second world war Germany faced a similar strain on resources as it penetrated Russia, who utilised its scorched land strategy to cement that strain onto advancing units.

Another logistical disparity is maritime supply \textendash\ responsible for further straining resources available within Germany. Early in the war, Germany goes under a British blockade, and despite German U-boats being heavily effective during the battle of the Atlantic (August 1914 - October 1918), they were eventually countered by the use of a convoy system. This meant that Germany was unable to import goods transported by sea. This thinning out of resources available to the Central Powers, as well as the previously discussed overwhelmed land transport meant that Germans could not sustain war, and despite the end of the two-front war in March 1918, the newly available army for the spring offensive was faced with the limitations of the country's strained logistics. This offensive, however, is notable for the invention of a tactic heavily used in the second world war \textendash\ tactical air support.

Despite the lack of success in the The Great War, tactical air support is central to the outcome of the second world war. Tanks supported by aircraft were highly used and effective in the German offensives in the Battle of France and overtaking the low countries (May-June 1940). Allied forces learned from early German successful tactics, and after the Normandy landings in June 1944, started to utilise the same. With the strain on German resources for once again fighting a two-front war, this strategy proved itself successful and led to German defeat less than one year later.

The use of aeroplanes was not wholly successful throughout the war, however it was very impactful to its outcome. The Battle of Britain (July 1940 - May 1941) was the theatre for the first battles involving only aeroplanes, and despite some destruction, destroyed cities would often rebuild in a matter of weeks, and the attacks did not force the British into an armistice \textendash\ its largest impact was increasing British morale, as well as being the first German defeat. On the pacific theatre, aeroplanes were being used for firebombing and eventually the atomic bombs, both incredibly important to the outcome of the war. Firebombings in Japan are thought to have killed between 240,000 and 900,000 civilians as well as displacing millions of others as they targeted urban areas. Nevertheless, much like in the western front, cities were often rebuilt within weeks, and so despite the worsening of the civil situation within Japan, it was thought they would not surrender and a land invasion might be required. Traditionalist historians argue that this thought culminated into the dropping of the atomic bombs on Hiroshima and Nagazaki on the 6th and 9th of August 1945, the definite conclusion to the war. 

In short, technological innovations were of great importance to both the first and second world war \textendash\ however from different perspectives. There are many important technological advances in the first world war, however it is argued that the most important are related to supply management and logistics, whilst on the second world war, the important developments are entirely based on the power of destruction of air raids, and the perfecting of tactical air support.

\bibliography{bibliography}
	
\end{document}
