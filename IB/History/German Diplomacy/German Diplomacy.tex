\documentclass{article}
\begin{document}
To what extent was German diplomacy to blame for the outbreak of war in Europe in 1939?


Taking over Austria and Sudentenland; allowing for Czechoslovakia to be broken up



In assessing the role of German diplomacy in the outbreak of the second world war, it is interesting to contextualise it with the nation's foreign policy as well as economic situation. As such, we must consider the two vastly distinct governments of the inter-war years: the Weimar republic (1918-33) and the Nazi regime(1933-39). The former is characterised by the crippling debt caused by war reparations, unemployment, American investment through the Young and Dawes plans, and eventual decline with the Great Depression; whilst the latter consisted of a boom in the German motor industry as well as the economy in general, almost extinction of unemployment and rearmament. The focus lies largely on the Nazi regime, whose economic prosperity had its foundation on the war industry. This meant, according to some historians, that Germany would soon lack civilian goods and an output for all the guns, artillery, etc, and being further exacerbated by its trade deficit, suggesting a very close link between their foreign policy and the country's economic needs.

Nazi foreign policy was heavily reliant on what historian William Craig called \textit{diplomatic smokescreen}, which coupled with the English appeasement policy, allowed for multiple conquests within Europe that required little military involvement and were met with no opposition. Following the diplomatic Munich Agreement in 1938 and the transfer of Sudetenland to Germany, there was an expectation that Germany would be appeased, but that was not the case. Czechoslovakia was suddenly split by territorial claims from Hungary and Poland, and a Nazi-backed independence movement in Slovakia. The F\"{u}rer's phoney diplomacy was clearly very effective in both territorial conquests and raising tensions within Europe.

Questioning the supremacy of France and United Kingdom. the European powers' expectation he occupation of Czechoslovakia  resulted in very little action from France or the United Kingdom \textendash\ showing the F\"{u}rer his phoney diplomatic means were highly effective.

Many actions that contradicted the post-war treaties, such as the remilitarisation of the Rhineland in March 1936, Anschluss in March 1938, and occupation of Sudetenland as well as portions of Czechoslovakia between October 1938 and March 1939, serve as examples of military actions which had been prefaced by some kind of diplomacy \textendash\ characterising the so-called diplomatic smokescreen.

 the occupation of Sudetenland, under the guise of pan-Germanism, came right after 

Indeed, Germany also used diplomacy for its protection through the Soviet-Nazi pact. Their intentions are made quite clear as the pact was signed on the 23rd of August 1939, and one week later Poland is invaded. Avoiding a two-front war has been a part of German policy since Bismarck . and outbreak of the war

Diplomacy is important insofar as it reflects Germany's economic needs (Raw material, heavily industrialised area of Rhineland and Sudentenland) and policies (Lebemsraum, unification with Austria, Pan-Germanism(Sudetenland, Austria, Saar and others))

\end{document}