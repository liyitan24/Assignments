\documentclass[10pt,a4paper]{article}
\usepackage{amsmath}
\usepackage[latin1]{inputenc}
\usepackage{amsfonts}
\usepackage{amssymb}
\usepackage[round]{natbib}
\usepackage{titlesec}
\newcommand{\sectionbreak}{\clearpage}
\bibliographystyle{abbrvnat}
\begin{document}
	
\begin{titlepage}
	\centering
	{\scshape\LARGE School \par}
	\vspace{1.5cm}
	{\huge\bfseries History Internal Assessment\par}
	\vspace{1cm}
	{\scshape\Large To what extent did the abolition of slavery contribute to the \\ proclamation of the republic in Brazil?\par}
	\vspace{2cm}
	{\Large Name\par}
	\vfill
	supervised by\par
	Name \textsc{Surname}
	
	\vfill
	
	% Bottom of the page
	{\large \today\par}
	{word count: 2025}
\end{titlepage}

\tableofcontents

\section{Source Evaluation}
	This study's research question is \textbf{``To what extent did the abolition of slavery contribute to the proclamation of the republic in Brazil?"}. The research question lends itself not to contrasting views, but to different approaches at looking at the same period. This paper, therefore, uses the sources to present multiple important aspects of the Republican movement and its relationship to the Abolitionist movement. It will be explored through two main sources:  Afr\^{a}nio Peixoto's Hist\'{o}ria do Brasil [History of Brazil], published in 1944, a broad historiographical work encompassing all of Brazilian history with a traditional historian approach; and Gilberto Freyre's Order and Progress, originally published in 1957, who focuses on the role of patriarchal society in the formation of Brazil, and specifically the transition to Republic of an empire that is also facing the slow transition from slave to free labour.
	
	
	Freyre was a polymath \textendash\,a sociologist and historian among many other things, who is best known for his anthropological historical accounts, wherein Order and Progress is one volume in the \textit{Casa Grande e Senzala} series discussing the Brazilian sociocultural formation, often through analyses of primary sources. The book was published in the beginning of Juscelino Kubitschek's democratic state, briefly after Get\'{u}lio Vargas' dictatorship (1930-45) \textendash\, a period which had been filled with doubts regarding its similarity to the Monarchy which had fallen decades earlier (1889) and Freyre's exile from Brazil. This suggests that the author might have been heavily influenced in his analyses by drawing parallels to the contemporary political situation of the country.  The book's purpose is to analyse the societal changes taking place in the country, i.e., end of slave trafficking, decentralisation of power, the role of the military, etc., emphasising the role of the people as a whole, which suggests he belonged to a marxist school of thought. Considering Freyre's attempts to prove his point throughout his writings, selection bias in the primary sources utilised is self-evident.  Nevertheless, Freyre's work is valuable for this investigation because it provides an interesting insight, both historical and sociological, into a ``clashing" Brazilian elite, i.e., both those who stood to benefit from the creation of a republic \textendash\, namely the upper echelon of the military \textendash\, and those who benefited greatly from slave labour, namely the rural aristocracy, as well as glimpses into the public mind. Moreover, because Freyre discusses from the monarchical period up to a far later period in the Republic, the depth one is able to achieve through his work, related to both the republican and abolitionist movements, is quite shallow. 
	
	
	Peixoto, on the other hand, was a doctor, politician, literary historian and writer member of the Brazilian Academy of Letters \textendash\,known for his biographies and works of fiction, and whose historiographical work is of far less impact than Freyre's. Hist\'{o}ria do Brasil's purpose is that of retelling the history of the country, often focusing on important characters, such as Princess Imperial Isabel and Gaston, Count of Eu \textendash\,taking a more Whig approach than Order and Progress. The content is often  narrative, and encompasses a broad historical scope, limiting the depth of the explorations. The book does not establish the traditional direct link between abolitionism and the republic, however, it  proves itself valuable by exploring the multiple causes to the decline of the Empire. It has its origins in a time of social struggle during the Second World War, though it does not transpire through the book. It does, however, attempt to be the complete opposite of apologist, with phrases such as ``It is sad to say, but history is not apology, and patriotism doesn't exclude bitter truths"\citep[p.225]{Afr44}. Although the book aims to explore the entire history of the country, it shows value by providing satisfactory insight into many of the consequences to the proclamation of the republic as well as the abolition of slavery. Another limitation is that the book dedicates itself to the Paraguayan War (1864-70) and its consequent wearing of the Empire, the latter being purposefully not explored in this paper due to the difficulty in comparing to the other reasons. Nonetheless, it describes in rich detail the context in which the abolitionist movement caused an immense labour shortage to the coffee industry in the South East. The book explores the other facets to the Republican movement gaining power since the early 1800s, including the Paraguayan War, the importance of the military and their rise to power and the lax reigning of Emperor Pedro II in the later years.
\section{Investigation}
	Brazil was the last country in the Americas to abolish slavery, and within the following year saw the change from an empire to a republic. The military coup d'\'{e}tat responsible for the affair was incredibly peaceful\citep[p. 7]{Frey86}, and in fact barely known by the general public. Nevertheless, historians often cite the abolition of slavery as the most important factor to the proclamation of the republic \citep[p. 60]{Brz10}. This Whig intentionalism, that is, the disconnect between an issue that pertains to the people and its resolution by a clique behind closed doors, has been seen throughout history and remains relevant to this day. Since the sources are complimentary to each other and have different \textendash\,though varying\textendash\,historiographical approaches, this essay attempts to create a full picture through the analysis of multiple vantage points, investigating the role of abolitionism in the change of form of government in Brazil, as well briefly analyse the role of the military, decentralisation of power and the republican movement itself.
	
%	\subsection{Abolition}
	Despite the often-cited bond between abolitionism and republicanism, Freyre discusses the existence of a ``token opposition" to the Republic by the recently freed blacks \cite[p. 8]{Frey86}. ``A deep-seated spirit of gratitude to the paternalistic monarchy" is how the author characterises the emergence of this counter-intuitive movement, wherein republican sympathisers were shooting monarchist negroes\cite[p. 9]{Frey86}, and it is clear to understand why they would put themselves in that position \textendash\, the movement had been elitist, the proclamation was unbeknownst to many\cite[p. 8]{Frey86}, and the Princess Regent was seen as personally responsible for the abolition, suggesting a very a strongly Whig historiographical view. Freyre's focus poses a stark contrast to Peixoto's, who by and large discusses the gap created by the lack of slave-labour and the existence of free-labour only in small parts of S�o Paulo \citep[p. 224]{Afr44}\textendash\, unusually marxist for the author. In fact, according to Peixoto, abolition was ``less in favour of the slaves than against the slaveowners"\citep[p. 225]{Afr44}, which cements the direct clash between the slavocrat rural elite and the Crown, explaining the empowerment of the anti-monarchical movement. Peixoto's unapologetic approach towards the Crown is very befitting of the time-period, in which the nation faced  Vargas' dictatorship (1930-45), bearing many similarities to the empire \footnotemark, and shines a very intentionalist light onto the events. This approach is made quite clear when describing the abolition: ``[It] costed the Redeeming Princess the throne, who paid for all. Brazil paid more"\citep[p. 225]{Afr44}. Interestingly, the private nature these decisions and events seemed to have suggest Freyre's Whig view to be the most ``correct".
		\footnotetext{Among others similarities, the exchange of political favours and power and empowerment of oligarchs.}
		
		
	 Freyre proceeds with his naturally functionalist focus, describing the people's move away from integration of Afro-Brazilian culture with the advent of republicanism \citep[pp. 10-12]{Frey86}, further distancing the movement from abolitionism. The white upper-classes had adopted an ``increasingly scornful" attitude towards the assimilation of portions of the Afro-Brazilian culture due to its use against republicans by the Crown-loyal blacks. It points out an interesting distinction between the various elites and upper classes which are present in Brazil, which although belonging to the same umbrella term of ``elite", share vastly different values: On one hand, there is the elite comprised by the military, who initially did accept aspects of Afro-Brazilian culture \citep[p. 11]{Frey86}, and on the other, there is the rural elite \textendash\, which can be further divided into slavocrats, who clashed with the Monarchy, and abolitionists, who employed European-imported free labour, etc.
	
%	\subsection{Local authority}
	The new political system would lead most Brazilians to seek economic and social stability with local leaders \citep[p. 230]{Frey86} \textendash\, a phenomenon which gave rise to coronelism\footnotemark, however the political power of this oligarchy predates the proclamation of the republic\citep{bieber1999power}. In fact, according to Bieber, the centralization of the power in the empire is to blame for the empowerment of the coron\'{e}is ��\textendash\, the rural elite, who by the 1850s were already recognised to be purchasing votes with glasses of cacha�a \footnotemark, and by the 1860s a proto-coronelism was already in place\citep[p. 198]{bieber1999power}. The power and importance given to municipalities meant that control of these affairs was, in practice, a strong influence in the political landscape of the nation as a whole. This rural oligarchy, largely slavocrat, was naturally highly against the abolitionist movement since its origins with the Free Womb Law \footnotemark, amongst others, and suffered great economical losses with the Lei �urea\footnotemark \textendash\, not only due to the loss of their slaves, but also labour shortages \textendash\, ``[The abolition] took place in the beginning of the coffee harvest, and for this reason was lost, ruined the crop in the Rio province". \citep[p. 224]{Afr44}
		\footnotetext[2]{System characteristic of the Old Republic (1889-1930) in which regional oligarchs, coron\'{e}is \textendash\, often members of the military or large farmers \textendash\, dominated politics through clientelism, an exchange of services and goods for political support.}
		\footnotetext[3]{A traditional spirit distilled from sugar-cane juice, similar to rum.}
		\footnotetext[4]{1871 law which considered all henceforth children of slaves to be free.}
		\footnotetext[5]{The Golden Law, signed by the regent Princess Isabel in 1888, abolishing slavery in Brazil.}
%\subsection{Republican movement}

	The coup d'\'{e}tat which formed the Republic was led by the military, however the republican movement had popular origins, and it did not attempt to have an abolitionist aspect to it. The 1870 Republican Manifesto published in the newly created newspaper \textit{A Rep\'{u}blica}, important for influencing the formation of the nation's first republican party Partido Republicano Paulista [S�o Paulo's Republican Party] in 1873, proposed to establish a federation based on the independence of the provinces \textendash\, forming states, ``linked by the same nationality and solidarity of greater interests of representation and external defence"\citep{ManifestoRep}, with mentions of the ``people's vote" and ``the people's sovereignty". Whether the word ``people" included slaves is naturally debatable, however both the republican manifesto and the republican party were spearheaded by a mixture of the intellectual elite \textendash\, doctors, lawyers, etc; abolitionists; and the S�o Paulo coffee farmers, among others, suggesting that Freyre's humanist marxist views are indeed correct, as far as the republican movement itself is concerned.
	
%\subsection{The military}
	The Paraguayan War had two discrete roles in the republican movement: establishing the power of the military and reducing the popularity of the Crown, though the latter will not be discussed in this paper. The military was historically a secondary institution \citep[pp. 60,61]{Brz10}, and the end of the war led to a larger degree of organisation as well as unity within the military, characterised by ``thinking, fanatically, those who died in campaign, would resurrect in glory" \citep[p. 223]{Afr44}. The war not only gave an important voice to republicans within the Armed Forces, but also allowed for the establishment of a ``sacred 'military honour'"\citep[p. xxxiv]{Frey86}, broadly used as a means to be above the law. Furthermore, the military had become largely conservative and aimed to ``slow down progress towards abolition" \citep[p. 10]{Par1996}. It meant, in principle, that by and large the military's involvement in both the coup and the republican movement were entirely self-serving, and with very little overlap with the abolitionist movement \textendash\, with notable statesmen of the Republic such as Joaquim Nabuco, famously an abolitionist, fervently believing replacing the monarchy was a mistake \citep[p. 100]{Frey86}.
	
%	\subsection{Conclusion}
	In summation, it is quite clear the republican movement did not have much overlap with the abolitionist movement, and neither did the elite at large, nor the upper echelon of military. Nevertheless, the Crown, weakened due to the Paraguayan war and giving continuation to its previous decades' abolitionist policies, clashed directly with a largely slavocrat rural elite, who in practical terms dominated politics. As the rural elite sought to support whichever political system benefited it the most, republicans found a new supporting group. The research, thus, suggests that the abolition of slavery greatly contributed to the formation of a republic in Brazil \textendash\, not due to the convergence of the two movements, but due to a political clash between the abolitionist regent and the slavocrat rural elite. This means that the direct correlation often made between the proclamation of the republic and abolition of slavery seems to be ``true", which shows that both Freyre and Peixoto seem to be ``correct", furthermore justifying the lack of clashing views on the issue, but rather attempts to explore the other accessory problems plaguing the Empire.

\section{Reflection}
This investigation has given me great insight into both the methodology used by historians and the challenges they face when attempting to evaluate sources in order to carry out an investigation. Often one is presented with diverging points of view \textendash\, especially when considering primary sources; other times, sources approach the same issue from completely different perspectives, and it becomes the historian's role to weigh these diverging approaches and perspectives in order to come up with a complete idea and conclusion which is closer to the ``truth". There is often a list of reasons cited as principal causes for the formation of the Brazilian Republic, so my role \textendash\, similar to that of a historian \textendash\, in this investigation was dealing with secondary sources, each focusing on a different issue, and attempting to weigh them in comparison to abolition of slavery. These diverging approaches led to another interesting challenge historians often face, which is trying to recognise an author's biases and their study's limitations. Gilberto Freyre's Order and Progress often cites primary sources, which of course is meant to improve his credibility, however, because he wants to discuss and emphasise the role of patriarchy in the formation of Brazilian society, he is prone to selection bias, presenting sources which will agree with the point he wants to make. It also raised the question of whether historians are meant to persuade or inform with their work \textendash\, a dilemma which Afr�nio Peixoto's Historia do Brasil, being more narrative than analytical, did not seem to face. Both Freyre's and Peixoto's works dealt with a far greater time period than just the relatively short time-frame before the Republic's conception, which in itself present a challenge in collection of information, but also highlights an interesting aspect of historians' methodology \textendash\, that of attempting to piece together bits of information from diverse origins in order to build a larger, satisfactory picture. With so much information that a historian could cover, deciding the scope of an analysis and how to keep it manageable becomes increasingly important. This investigation showed the wealth of knowledge they are required to have in order to wade through all the information and select the few they deem more important \textendash\, made clear by the omission of the decline in popularity of the Crown, for instance. In turn we are faced with yet another question: Where is the line between selection by relevancy, bias and omission?
\nocite{ManifestoRep}
\bibliography{bibliography}
\section{Apendix - Translations}

\citep[p. 60]{Brz10} Many factors contributed to the fall of the Empire, however, among them is a fundamental factor: The abolition of slavery in 1888.

\citep[pp. 60,61]{Brz10} Only after the Paraguayan War, the Brazilian military, up to that point recognised as a secondary institution, becomes conscious of its power and importance to the country. Furthermore, they begin to organise themselves politically and express political opinions, among them the defence of republicanism.

\citep[p. 222]{Afr44} Of this campaign, which we were provoked and had to support an indispensable alliance of Argentines and Uruguayans, we operated acts of bravery, suffered deprivations and epidemies, spent half a million contos de r\'{e}is [conto de r\'{e}is is one thousand r\'{e}is], wasted on the Prata [Rio de la Plata], without national economic compensation, and sacrificed one hundred thousand brazilians, broadly voluntaries, sometimes recruited

\citep[p. 224]{Afr44} Materially, a disgrace: we didn't have but in a portion of S\~{a}o Paulo free labour. It took place in the beginning of the coffee crop, and therefore ruined the crop in the province of Rio, the most important at the moment, with this theft, property guaranteed by the State, spoiled without reparation or assistance to the free negroes, who, abandoned and incapable of winning life [earning money], invaded the cities and the capitals.
\end{document}