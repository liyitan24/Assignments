\documentclass{article}

\usepackage[utf8]{inputenc}
\usepackage[danish]{babel}
\usepackage[round, comma, authoryear]{natbib}
\bibliographystyle{unsrtnat}


\begin{document}
\section{Rationale}

Opgaven er en udskrift af en kort del af en fiktiv telefonsamtale mellem Hassan og hans kæreste May, hvori han klager til May, om konferencen han deltager i. Genren Hassan baseres på hovedpersonen i Hassan Preislers ``Brun mands byrde'', især hans sarkasme og måde at latterligegøre de ulige situationer han beskriver.

Skrivestilen prøver at replikere Hassans egen, gennem brug af nutid, selvom han taler om fortiden \textendash\ ``jeg kommer til konferencen i Salerno''; sjove, ukonventionelle kendetegn af andre, som for eksempel ``han kigger på mig med skægget og de hvide tænder''; og selvom det ser ud til, at teksten mangler struktur, er Preislers spontanitet jeg prøver at genskabe: Blandt andet, komma bruges tit i stedet for punktummer, ``og'' bruges på en repetitiv måde: ``så kommer du med kjolen og hestehalen og lyst at danse, og redder mig fra samtalen''. Den valgte genre passer rigtigt godt til at efterligne den beskrevne stil.

word count: 153

\section{Assignment}

HASSAN Hej, min May

MAY Hej. Hvordan var flyveturen til Salerno? og Konferecen?

HASSAN Den var fint. Denne gang holdes konferencen i et lille hotel, og det er fuld af politiker og direktører og konsulenter, og de kommer til mig efter min tale, og med en hånd ryster de min hånd og siger ``\textit{good job}'', og med den anden holder de skrøbelige papirstallerkner fuld af carciofi alla romana, ravioli og parmigiana.

MAY Nej\ldots

HASSAN De har sølv hår, dyre jakkesæt på, og selvom de sidder og jeg står op, er de et hoved højere end mig, og de inviterer mig at sidde ned. De høje herrer taler om mangfoldighed, og de nyitaliener, nyenglænder og nydansker, og interkulturel baggrund og integration, og de kigger på mig med skægget og de hvide tænder og brillerne for at lede efter godkendelse, og jeg nikker. ``I må væbne jer med tålmodighed'' siger de, imens de klapper hinanden på ryggen, og det ligner nytårsfesten hvor du bliver inviteret af de tvillinger Andreas og Mads\ldots

MAY Nå, ja. Det kan jeg huske.

HASSAN \ldots Ja? jeg bliver inviteret af Aner Det Ikke, og selvom vi kommer til fest, tvillingerne tænker at det er et interview med mig, og så spørger de ``hvor kommer du fra?'' \textendash\  hjemmefra. ``hvor er det?'' \textendash\ i Gentofte. ``hvor blev du født?'' \textendash\ Gentofte hospital. Og så de endeligt spørger ``hvad er dine forældre?'' og jeg siger ``lærer og kardiolog'', så kommer du med kjolen og hestehalen og lyst at danse, og redder mig fra samtalen,  og vi danser til ``Det er ganske enkelt'' og ``Det regner i mit hjerte'', og taler om dit liv som sygeplejeske og mit liv som skuespiller, og natten slutter og jeg bliver forelsket i dig, og du kan næsten ikke huske mit navn, og sådan er vi ikke længere.

MAY Hvor sødt, skat. Sikkert er der noget godt der sker der.

HASSAN Altså \ldots Inklusions- og integrationsseminarer stinker af smørsyre fra politikernes sokker, der prøver at sælge sjælen til civilisterne, og måden de overvinder det, er at spraye alt med den nyeste nye: fusionbuffeter, og tvetydigt oprindelse, \textit{cool-chic} tøj, der råber ``jeg forstår dig'' og ``stem på mig'', og de inviterer de andre taler fra Tyskland og England og Sverige, og vi allesammen hedder Hassan, og de gentager ``det er så vigtigt I kunne komme'', og de har høj hat og monokel på, og de er en \textit{golden gun} væk fra James Bond, der bryder gennem muren, og jeg håber det sker, så vi kan allesammen rejse os og råbe ``viva la revolución'' på vores egne sprog, men det sker ikke, og vi vover ikke, så i stedet for, når en Hassan bliver færdig med talen, så trækker han vejret og stiger langsomt ud, og alle klapper, og vi dykker ansigtet i iPhones eller aviser, og ønsker bare at blive færdig med seminaret.

MAY [grinende] Hassan\ldots

word count: 479

\nocite{Preisler2013BMB}
\bibliography{bibliography}
\end{document}
