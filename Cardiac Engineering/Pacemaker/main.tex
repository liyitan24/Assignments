\documentclass{article}


\begin{document}

\section{Introduction}
The following properties have been determined as important for the most appropriate wire:
\begin{itemize}
	\item Diameter less than 2mm
	\item High flexibility
	\item Biocompatible
\end{itemize}

\section{Procedure}
The equipment necessary is as follows:
\begin{itemize}
	\item Vernier Caliper
	\item Ruler
	\item Wires to be tested
\end{itemize}
Utilise the caliper to determine the wires' diameter.

In order to determine the wire's flexibility, we utilise the following method:
	
Lay the wire under test on a horizontally and perpendicular to an edge, then push it forward until a certain length $l$ extends over the edge. We then measure the vertical deflection, $\delta_{max} $. The flexibility, $c$ can be determined by
	\begin{equation}
	c = \frac{\delta_{max}}{l^4}
	\end{equation}		

\section{Raw data}
The wires of appropriate dimensions and material are as follows
\begin{table}[h]
	\begin{tabular}{c|c|c|}
		Diameter & Coating & Flexibility ($cm^{-3}$) \\
		1.10 & Polyurethane & 0.001544 \\
		1,90 & Silicon & 0,001749
	\end{tabular}
\end{table}
\section{Analysis}
Both polyurethane and silicon are known to be long-term biocompatible, therefore a final solution may come down to the individual cost of the wiring \textendash something beyond the scope of this report.
\end{document}          
