\begin{figure}[h]
\begin{minipage}{0.48\textwidth}
    \centering
    \begin{tikzpicture}[scale=0.85]
	    \begin{semilogxaxis}[
	    	title={PBS solution}, xlabel=Frequency $(\si{\hertz})$, ylabel=Amplitude $(\si{\milli\volt})$, ymin=0, ymax=130, %axis x line = bottom, axis y line=left,
	    ]
	    	\addplot[] table{PBS.dat};
	    \end{semilogxaxis}
    \end{tikzpicture}
\end{minipage}
\begin{minipage}{0.48\textwidth}
	\centering
	\begin{tikzpicture}[scale=0.85]
		\begin{semilogxaxis}[
			title={Yeast 20g/L solution}, xlabel=Frequency $(\si{\hertz})$, ylabel=Amplitude $(\si{\milli\volt})$,
			ymin=0, ymax=130,
		]
		\addplot[] table{20gLGainOf100.dat};
		\end{semilogxaxis}
	\end{tikzpicture}
\end{minipage}
\caption{Amplitude based on frequency. Note the dip at \SI{50}{\hertz} due to the filter.}
\end{figure}

\paragraph{Assess performance of reactor prototype; cell density with final design}
\paragraph{Uncertainties} Consider trends vs values? Probably most important for the calibration curve.
\paragraph{important comments on results, relate to theory and knowledge}
\paragraph{Limitations and improvements}