\begin{enumerate}
	\item 
	\item 
	\item 
\end{enumerate}
\paragraph{Variables}
\begin{itemize}
	\item Control:
		\begin{itemize}
			\item Output voltage on function generator: 10V.
			\item Probes 180º from each other. Sense leads 30º from the probe and 180º from each other. See Figure \ref{fig:ElectrodeOrientation}.
			\item Gain $\frac{R_3}{R_2} = \frac{15k}{150} = 100$.
			\item Liquid volume 250 ml, 60mm height measured from inside the cup.
		\end{itemize}
	\item Independent: Frequency.
	\item Dependent: Voltage peak-to-peak
\end{itemize}

<<<<<<< HEAD:Engineering Challenge 2/files/Lab1.tex
\paragraph{Considerations}

Due to a limited power input into the amplifier, a gain ratio of 1000 led to \textit{gain saturation}, where we no longer observed the expected amplification.

Although we define probe orientation, there are two factors we are disregarding: the cup is not perfectly cylindrical, and the leads are not straight -- so the distance between electrodes is smaller than what the experimental setup implies.

With gain of 10, the maxima appeared in different frequencies as to those of 100 and 1000. Suspect to be related to the percentage error being too large.

\subsection{Raw data}
\begin{figure}[h]
\begin{minipage}{0.48\textwidth}
    \centering
    \begin{tikzpicture}[scale=0.8]
	    \begin{semilogxaxis}[
	    	title={PBS solution}, xlabel=Frequency $(\si{\hertz})$, ylabel=Amplitude $(\si{\milli\volt})$, ymin=0, ymax=120,
	    ]
	    	\addplot[] table{PBS.dat};
	    \end{semilogxaxis}
    \end{tikzpicture}
    \caption{Amplitude based on frequency. Note the dip at \SI{50}{\hertz} due to the filter.}
    \label{fig:}
\end{minipage}
\begin{minipage}{0.48\textwidth}
	\centering
	\begin{tikzpicture}[scale=0.85]
		\begin{semilogxaxis}[
			title={Yeast 20g/L solution}, xlabel=Frequency $(\si{\hertz})$, ylabel=Amplitude $(\si{\milli\volt})$
		]
		\addplot[] table{20gLGainOf100.dat};
		\end{semilogxaxis}
	\end{tikzpicture}
	\caption{}
	\label{}
\end{minipage}
\end{figure}


=======
>>>>>>> 2544e85dfba9b8c2a4cb72a9d48d7796db5bc51d:Engineering Challenge 2/files/Procedure.tex
\begin{figure}[h]
	\newcommand{\Radius}{3}
	\newcommand{\RadPow}{0.5}
	\newcommand{\RadVolt}{0.5}
	\centering
	\begin{tikzpicture}[scale=0.85]
		\draw (0,0) circle (\Radius cm);
		%Driving electrodes
		\draw [dashed,cyan] (0,-\Radius) node{*} -- (0,\Radius) node{*};
		\node at (0.3,-\Radius-0.3) {-$V_x$};
		\node at (0.3,\Radius+0.3) {$V_x$};
		\draw [cyan] (0,\Radius) -- ++ (0,\RadPow)
			-- ++ ({-\Radius*1.3},0)
			-- ++ (0,{-\Radius-\RadPow)}) 
			++ (0,-\RadPow) circle (\RadPow) node{\mbox{\fontsize{25}{21.6}\selectfont\( \sim \)}} ++ (0,-\RadPow)
			-- ++ (0,{-\Radius+\RadPow})
			-- ++ ({\Radius*1.3},0)
			-- ++ (0,{\RadPow})
		;
		% Sensor electrodes
		\draw[dashed, red] ({\Radius*cos(60)},{-\Radius*sin(60)}) node{*} -- ({-\Radius*cos(60)},{\Radius*sin(60)}) node{*} coordinate (a);
		\draw [red]
			(a) -- ++ (-{(2*\RadVolt*0.8)*cos(60)}, {(2*\RadVolt*0.8)*sin(60)}) coordinate (a)
			-- ++ (0,0) arc (-60:120:0.2)
			-- ++ ({1.3*\Radius*cos(30)},{1.3*\Radius*sin(30)})
			-- ++ ({(\Radius+\RadVolt)*cos(60)},{-(\Radius+\RadVolt)*sin(60)})
			++ ({\RadVolt*cos(60)},{-\RadVolt*sin(60)}) circle (\RadVolt) node{V} ++ ({\RadVolt*cos(60)},{-\RadVolt*sin(60)})
			-- ++ ({(\Radius+\RadVolt)*cos(60)},{-(\Radius+\RadVolt)*sin(60)})
			-- ++ ({-1.3*\Radius*cos(30)},-{1.3*\Radius*sin(30)})
			-- ++ ({(-2*\RadVolt*0.8)*cos(60)}, {(2*\RadVolt*0.8)*sin(60)})
		;
		%
		\draw [<->] (0,\Radius*0.5) arc (90:120:\Radius*0.5) node[pos=0.5,anchor=south]{30º};
		\matrix [draw, matrix of math nodes] at (current bounding box.north east) {
    		 \color{cyan}* & \text{Driving electrodes} \\
			 \color{red}* & \text{Sensing electrodes} \\
	 	};
	\end{tikzpicture}
	\label{fig:ElectrodeOrientation}
	\caption{Orientation of electrodes}
\end{figure}

\paragraph{Considerations}

Due to a limited power input into the amplifier, a gain ratio of 1000 led to \textit{gain saturation}, where we no longer observed the expected amplification.

Although we define probe orientation, there are two factors we are disregarding: the cup is not perfectly cylindrical, and the leads are not straight -- so the distance between electrodes is smaller than what the experimental setup implies.

With gain of 10, the maxima appeared in different frequencies as to those of 100 and 1000. Suspect to be related to the percentage error being too large.

Maximum amplitude for PBS around 3000hz, maximum amplitude for yeast at 5000hz. We pick 4000 Hz.
There is no concern for cell disruption at the range of frequencies we are analysing (Yerworth, R., 2019 November 28)

Uncertainty of about +/- 2ml with our beakers. Relevant in the construction of the calibration curve.

We are creating the calibration curve without the impeller. Because it's plastic, we assume very little effects in the condutance

The frequency we inputed was also observed in mydaq's oscilloscope. 