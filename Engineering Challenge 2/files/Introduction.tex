% \paragraph{Background information on the topic.}

% \paragraph{State the problem, vision and purpose}
The aim of the experiment is to establish a calibration curve against which we can track and compare the progress of our bioreactor.
We do so by defining the appropriate frequency for best potential difference output on a phosphate buffer solution, as well as a 20 g/l yeast solution.
Finally, we produce the calibration curve by analysing the potential difference at a range of yeast concentrations.  

\paragraph{Background knowledge.}
Impedance spectroscopy is a widely used modality to establish cell composition, diagnosis of lymphatic edema and, most importantly for our purposes, quantify biological cell suspensions \cite{LvovichVadimF2012Is, PolatAyfer2017EDoL}.
Alternatively optical density can be used, which we in fact do use to verify our results.

Impedance (Z) is effectively resistance in an AC electrical circuit, which means that it follows Ohm's Law.
Furthermore, we can represent a cell in extracellular liquid as capacitor in a electrical circuit, which allows us to use an important definition from AC theory; where V is potential difference and I is current, $C$ is capacitance, $f$ is the frequency, $j$ is the imaginary unit:
\begin{equation}
    V = ZI \quad \quad \quad \quad Z_c = \frac{-j}{2\pi f C}
\end{equation}

In practice, we cannot determine $C$, but by combining both equations, we know the following relationship
\begin{equation}
   V_{cell} \propto \frac{1}{f}
\end{equation}

Which tells us that as we increase the function generator's frequency, the potential difference ``lost'' to the cells is reduced, and therefore, the potential that we measure in the extracellular liquid is higher.
\begin{equation}
    V_{liquid} \propto f
 \end{equation}