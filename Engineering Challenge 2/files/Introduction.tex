% \paragraph{Background information on the topic.}


Aim of the experiment is to establish a calibration curve against which we can track and compare the progress of our bioreactor.
% \paragraph{background knowledge}
% \paragraph{State the problem, vision and purpose}
% \paragraph{Background knowledge, status quo, technical problem, competing solution}
Impedance spectroscopy is a widely used modality to establish cell composition, diagnosis of lymphatic edema and, most importantly, biological cell suspensions \cite{LvovichVadimF2012Is, PolatAyfer2017EDoL}.
Alternatively optical density can be used, which we in fact do use to verify our results.

Impedance (Z) is effectively resistance in an AC electrical circuit, which means that it follows the same Ohm's Law definition; where V is potential difference and I is current:
\begin{equation} \label{eq:Z}
    Z = \frac{V}{I}
\end{equation}

We can represent a cell in extracellular liquid as capacitor in a electrical circuit, which allows us to use an important definition from AC theory; where $C$ is capacitance (the quality of a capacitor), $f$ is the frequency, $j$ is the imaginary unit.
\begin{equation} \label{eq:Z_c}
    Z_c = \frac{-j}{2\pi f C}
\end{equation}

In practice, we cannot determine $C$, but by combining equations (\ref{eq:Z}) and (\ref{eq:Z_c}), we know the following relationship
\begin{equation}
   V \propto \frac{1}{f}
\end{equation}

Which tells us we can alter a function generator's frequency such that the voltage output from the ``cell as an electrical circuit'' is the most meaningful, telling us information about its impedance, and consequently the cell growth.

\textbf{Why do we run PBS?} comparison to cell 20g/l
\textbf{why does an increase in frequency give us better results? theory says otherwise}