Risk analysis and mitigation of hazards is a priority that greatly influence the time scale and feasibility of the project as whole. Several risks arise both physically and economically, hence at the detriment of Uganda’s economic development which must be evaluated.  

Financial risk must be investigated. To what extent can the size of the plant be pushed as to not cause harm to the country’s budget?  The cost should be calculated from real data and the costs of similar projects. If required, the World Bank can help fund this project as well GAVI, an enterprise dedicated to providing immunisation coverage.  

Moreover, local university students can be trained to tend to the bioreactor. This solution allows for employment of educated graduate with no incentive to leave and little bargaining power for high wages. Furthermore, local employment increases the overall sustainability of the project through promoting a circular economy.

Sustainability of the whole project plays a vital role, especially in convincing the stake holders to invest. Multiple stake holders will be affected by this project, and one of them are the local farmers who need to sell their land to allow the facilities to be built. This will affect their economy which heavily depends on farming, but the diversification of the economy, with increased manufacturing and services, provide job alternatives, will contribute to local employment. If the project can be built solely by local construction companies,  population and the state will be further benefited, as the facilities should ultimately be cheaper. 

Although we do not aim to vaccinate the entire population in the first year, the future has been considered – largely focusing on either targeting neighbouring nations or other critical maladies, such as cholera. In 2018, 45 Democratic Republic of the Congo refugees were killed and over 2000 others were hospitalized due to Cholera infection \cite{WHO2018}. This issue, if not address properly, will cause a disaster, not only in terms of physical health but also the economy in general as for every \$1 spent on vaccines, \$16 are saved on administering treatment \cite{GaviTheVaccineAlliance2019}. 

The current education system in Uganda is concerning as only 40\%  of the students are literate after graduating from primary school while only 24\%  of adolescents are enrolled in a secondary school \cite{UNICEF&GovernmentOfUganda2015}. Being physically healthy allow a person to prosper in their education and improve their metacognitive skills. This vaccination program will increase educational attainment hence creating a better workforce for the country in the future. 