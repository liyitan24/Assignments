We expected to see medium-strain aerobic exercises leading to a much higher pulse rate than the one achieved those of anaerobic.
Seeing as this was not the case, the comparison was abandoned.

Rest heart rates were measured the moment before each new set.
Nevertheless, while measuring heart rate, sitting down or squatting meant a drop of around 20 BPM, and we are unsure whether Subject 2's measurements were taken sitting down – or their increased level of fitness, as expected, accounts for the faster recovery rate.
Note that subject 1 is out of shape, and expectedly has a higher heart rate during the 2 or so minute rest period, however it is imperative that this is controlled for in further experiments.

Despite the expectation that vasoconstriction on fingers would show a reduction in oxygen saturation, the largely stable blood oxygenation suggests otherwise.
Given that the weights are held in one's hand, this may explain the high oxygenation when we would otherwise expect blood flow and oxygen to be largely focused on the larger muscles performing the exercise.

For subject 1 muscular fatigue was accompanied by a considerably raise in the barbell's mass due to time constraints.
An increase in the exercise's strain would lead to higher pulse rate, and therefore this study is unable to differentiate between increased strain and pure fatigue.
A further study would be required forcing fatigue at ``medium'' weights with more sets, for an appropriate comparison.

The attempt at having one female and one male subject was appropriate, and a larger sample size for a future experiment would be ideal.
This means having access to a larger group of fit and unfit individuals who are well versed with this particular exercise in order to push for muscular fatigue; the latter of which is likely to be a problem.
Ultimately, each subject's physiological responses are only measurable against themselves, but they provide information on variability.

Conclusively, all the hypothesis were incorrect.
Given our understanding of improved heart recovery post exercise and the results, this is perhaps the best conclusion that can be drawn from this experiment.
