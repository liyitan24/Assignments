\subsection{a}
\lstinputlisting[caption={Topic 9/10. Question 4. a}]{"./files/topic9/a.m"}

(i) $2.8 \cdot 10^{-3}$

(ii) $1.4901 \cdot 10^{-4}$

(iii) Assuming one bit is $8\cdot 10^6$ megabytes, then
\begin{equation*}
    \text{time} = \frac{1.345 \cdot 8 \cdot 10^{6}}{370 \cdot 60 \cdot 60} = 2.9081 \cdot 10^4 \si{\second}
\end{equation*}
Which is approximately 8h 47min

\subsection{b}


(i)
\begin{equation*}\begin{split}
    \mu(1-10) = 1160; \quad \sigma(1-10)=288.27 \\
    \mu(11-20) = 721.8; \quad \sigma(11-20)=302.37 \\
\end{split}
\end{equation*}

(ii) Years 1-10: 953.8 to 1366.2; Years 11-20: 505.5 to 938.1

\lstinputlisting[caption={Topic 9/10. Question 4. b}]{"./files/topic9/b1.m"}

(iii) Given the results from (ii), the upper band for rainfall in years 11-20 fall strictly under the minimum for years 1-10.
Alternatively, for the average rainfall of years 11-20 to be larger than the 1-10 average, it must be 2 standard deviations away or above -- meaning there's about 97\% chance its value is lower than mean for years 1-10, 1160.